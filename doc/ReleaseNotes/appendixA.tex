This appendix describes changes introduced into MODFLOW~6 in previous releases. These changes may substantially affect users.

\begin{itemize}
	\item Version mf6.2.2--July 30, 2021

	\underline{NEW FUNCTIONALITY}
	\begin{itemize}
	        \item A new Adaptive Time Step (ATS) utility was added.  The ATS utility allows any stress period to be overridden with an alternative time stepping approach.  The ATS utility implements two main capabilities (1) the capability to retry failed time steps with a shorter time step repeatedly until convergence is achieved, and (2) the capability to shorten and lengthen time steps based on simulation behavior.  These capabilities are described in the user input and output guide in a new section on the ATS utility.
	        \item A new option for printing water contents to a dedicated output file has been added to UZF.  To activate, the keyword WATER\_CONTENT is added to the OPTIONS block of UZF, followed by FILEOUT, followed by the user-specified output file name, for example ``water-content.uzf.bin''.  The approach is analogous to the STAGE option within the SFR options block.  Contents of the new file will be written in binary and can be read using flopy's binaryfile utility.  
	        \item The residual balance error for groundwater flow and solute transport is now written to the diagonal position of the flowja array, which is marked with the text description ``FLOW-JA-FACE''.  The flowja array is optionally written to the binary model budget file according to user settings in the output control file and other package input files.
	        \item A new option for simulating specific storage changes only when a cell is fully saturated has been added to the storage (STO) package. To activate, the SS\_CONFINED\_ONLY keyword is added to the OPTIONS block in the STO Package. This option is identical to the approach used to calculate storage changes under confined conditions in MODFLOW-2005.
	        \item A new observation type called ``wel-reduction'' was added for the Well Package.  This observation type reports the reduction in the well discharge that can occur when the \texttt{AUTO\_FLOW\_REDUCE} option is specified.
	\end{itemize}
	
	\underline{EXAMPLES}
	\begin{itemize}
	        \item Added the following new examples: 
	        \begin{itemize}
	          \item ex-gwt-hecht-mendez
	          \item ex-gwf-capture (This example is described in mf6examples.pdf to demonstrate functionality of the Application Programming Interface; it is not included in the examples folder of this distribution as it requires python and several python packages)
	        \end{itemize}
	        \item Added new citation to this document.  The \cite{morway2021} paper describes the use of the Water Mover Package in MODFLOW~6 to represent natural and managed hydrologic connections. 
	\end{itemize}

	\textbf{\underline{BUG FIXES AND OTHER CHANGES TO EXISTING FUNCTIONALITY}} \\
	\underline{BASIC FUNCTIONALITY}
	\begin{itemize}
		\item The specific storage formulation in the storage (STO) package has been modified to eliminate the dependency of the original formulation on the vertical datum. The original specific storage formulation also overestimated storage changes for cells that resaturated or desaturated in successive time steps. Furthermore, the sign of the specific storage change was incorrect in cells with negative heads and resaturated or desaturated in successive time steps. The revised specific storage formulation resolves all of the deficiencies of the original formulation and accurately simulates specific storage changes under water table conditions but will change the results for existing models. Testing indicates that the differences between models run with the original and revised specific storage formulation are generally small but tend to increase in models with large specific storage values or have cells that repeatedly resaturated or desaturated in successive time steps.
		\item The convergence failure message message written to GWF and GWT listing files (FAILED TO MEET SOLVER CONVERGENCE CRITERIA) is now written after the budget summary tables.  In previous releases this convergence failure message was written prior to printing heads and concentrations, which often resulted in this message being unnoticed by users.
	        \item The order of output written to the GWF and GWT listing files for a time step was reorganized in a consistent manner with model and package flows coming first, followed by dependent variables, and then concluding with budget summary tables.
	        \item The DISU Package checks to make sure that the top of a cell is not higher than the bottom of an overlying cell.  A new option was added to the DISU Package to allow the user to specify the vertical offset tolerance used in this check.  This new optional input variable is VERTICAL\_OFFSET\_TOLERANCE.
	        \item Add DISU Package check to ensure that JA(IA(n)) is equal to n and that no values in JA are less than zero or greater than nodes.
	        \item When IDOMAIN is used with the DISU Package and any IDOMAIN value is zero, then the program was expecting all JA values to be positive. The program is supposed to allow a negative JA value to be specified for the corresponding cell (in the diagonal position), but this was not working.  A fix was implemented to allow a negative cell number to be specified in the diagonal position of the JA array when the IDOMAIN capability is active.
	        \item A new check was added to the Horizontal Flow Barrier (HFB) Package to ensure that barriers are between cells that are horizontally connected.  The program would previously continue running if a barrier was between vertically connected cells.
	        \item There was no check to prevent the zero-order decay functionality of the Mobile Storage and Transfer (MST) and Immobile Storage and Transfer (IST) Packages in the GWT Model from producing negative concentrations.  The program now reduces the zero-order decay rate for the aqueous and sorbed phases (for the mobile and immobile domains) to ensure that decay does not consume more mass than is available.  These changes do not affect zero-order growth.
	        \item If a binary budget file from a GWF Model was larger than about 2 Gigabytes, then it could not be used as input for a subsequent GWT Model.  The program was modified to use a long integer to store the byte position.
	        \item The program was terminating with a non-zero return code if the simulation did not converge.  This is the intended behavior, unless the CONTINUE option is specified in the simulation name file.  The program now terminates with a return code of zero if the simulation does not converge, but the CONTINUE option is set and the program reaches the end of the simulation.
	\end{itemize}

%	\underline{STRESS PACKAGES}
%	\begin{itemize}
%	        \item xxx  
%	        \item xxx  
%	        \item xxx  
%	\end{itemize}

	\underline{ADVANCED STRESS PACKAGES}
	\begin{itemize}
	        \item The UZF water-content observation by depth was giving an error, because a check was using the wrong index to retrieve the cell top and bottom elevations for the requested observation.  The program was modified to use the correct index, and the output is now as expected.  Note that this bug is not related to the new WC keyword in the OPTIONS block, but rather is related to OBS6 output option.
	        \item Amend surfdep error check with landflag.  Deep cells (non-land surface cells) should not require surfdep > 0
	        \item In the LAK observation package, users can specify ``lak'' to get a summary of lake-groundwater exchange.  Users could specify a lake number without specifying a specific connection number (variable ``iconn'').  Code will now stop if lake number is provided without a matching connection number.  Code will still provide a summary of total lake-groundwater exchange when BOUNDNAME is entered for the variable ID.  This also will fix a similar issue for the observation types ``wetted-area'' and ``conductance'', since both require ID2 when ID is an integer corresponding to a lake number.
	        \item In the MAW observation package, users can specify ``maw'' to get a summary of well-groundwater exchange.  The code was allowing users to specify a well number without requiring specification of a connection number (variable ``icon'').  Code will now stop if well number is provided without a matching connection number.  Code will still provide a summary of total well-groundwater exchange when BOUNDNAME is entered for the variable ID.  This also will fix a similar issue for the observation type ``conductance'', since both require ID2 when ID is an integer corresponding to a well number.
	\end{itemize}

	\underline{SOLUTION}
	\begin{itemize}
	        \item An optional new input variable called ATS\_OUTER\_MAXIMUM\_FRACTION can now be entered for the IMS Package.  This variable has no effect unless the new ATS capability is active.  
%	        \item xxx  
%	        \item xxx  
	\end{itemize}

\end{itemize}



\begin{itemize}
	\item Version mf6.2.1--February 18, 2021

	\underline{NEW FUNCTIONALITY}
	\begin{itemize}
	        \item The Source and Sink Mixing (SSM) Package for the Groundwater Transport Model was modified to include an alternative option for the concentration value assigned to sinks.  A new AUXMIXED option was added to represent evaporation-like sinks where the solute or a portion of the solute may be left behind.  The AUXMIXED option provides an alternative method for determining the groundwater sink concentration.  If the cell concentration is larger than the user-specified sink concentration, then the concentration of the sink will be assigned as the specified concentration.  Alternatively, if the specified concentration is larger than the cell concentration, then water will be withdrawn at the cell concentration.  Thus, the AUXMIXED option is designed to work with the Evapotranspiration and Recharge packages where water may be withdrawn at a concentration that is less than the cell concentration.  
	        \item Add support for the Freundlich and Langmuir isotherms to the Mass Storage and Transfer (MST) Package of the Groundwater Transport Model.
	\end{itemize}
	
	\underline{EXAMPLES}
	\begin{itemize}
	        \item Added the following new examples: 
	        \begin{itemize}
	          \item ex-gwt-mt3dms-p02
	          \item ex-gwt-rotate
	          \item ex-gwt-saltlake
	          \item ex-gwt-uzt-2d
	        \end{itemize}
	\end{itemize}

	\textbf{\underline{BUG FIXES AND OTHER CHANGES TO EXISTING FUNCTIONALITY}} \\
	\underline{BASIC FUNCTIONALITY}
	\begin{itemize}
	        \item The way in which the dispersion coefficients are calculated with the simple option (XT3D\_OFF) for the Dispersion Package was modified.  When the velocity within a cell is not aligned with a principal grid direction, the  dispersion coefficients are calculated using a simple arithmetic weighting, rather than harmonic weighting as is done for the simple option for anisotropic flow in the NPF Package.  The arithmetic weighting option eliminates a possible discontinuity when a principal flow-aligned dispersion component is zero.
	        \item The mass flow between two cells is calculated and optionally written to the GWT budget file.  There was an error in this calculation of mass flow when the TVD scheme was specified in the Advection (ADV) Package.  Consequently, the mass flows written to the budget file were not correct in this situation.  Because these mass flows are also used in the budget calculations for the Constant Concentration (CNC) Package, reported CNC mass flows were also not correct.  This could result in large budget discrepancies in the GWT budget table.  Simulated concentrations were not affected by this error.  A small correction was made to the routine that adds the advective mass flow for the TVD scheme.
	        \item Several packages had input blocks that could not be specified using the OPEN/CLOSE keyword.  The program was modified so that OPEN/CLOSE is supported for all intended blocks.
	        \item The Immobile domain Storage and Transfer (IST) Package for the GWT Model is based on a conceptual model in which the immobile domain is always fully saturated, and so the saturation of the immobile domain does not depend on head in a cell.  The program was modified so that none of the immobile domain terms include saturation, except for the mass transfer equation itself, in which the transfer of solute between the mobile and immobile domain is multiplied by the cell saturation.
	        \item Budget terms for the Immobile domain Storage and Transfer (IST) Package were not being written to the binary budget file for the GWT Model.  The package was modified to write these rate terms to the GWT binary budget file using the settings specified in the GWT Output Control file.
	        \item Bulk density does not need to be specified for the Immobile domain Storage and Transfer (IST) Package if sorption is not active; however the program was trying to access bulk density even though it is not needed, which resulted in an access violation.  Program was fixed so that bulk density does not need to be specified by the user unless sorption is active for the IST Package.
	        \item Budget tables printed to the listing file had numeric values that were missing the `E` character if the exponent had three digits (e.g. 1.e-100 or 1.e100).  Writing of the budget table was modified to include the `E` character in this case.  This change should make it easier for programs written in other languages to parse these tables.
	        \item In the Mass Transfer and Storage (MST) and the Immobile Storage and Transfer (IST) Packages, the keyword to activate sorption was changed from SORBTION to SORPTION.  The program will still accept SORBTION, but this keyword will be deprecated in the future.
	        \item Revised several of the text strings written to the headers within the GWT binary budget file.  A table of possible text strings for the GWT binary budget file are now included in the mf6io.pdf document.
	\end{itemize}

%	\underline{STRESS PACKAGES}
%	\begin{itemize}
%	        \item xxx  
%	        \item xxx  
%	        \item xxx  
%	\end{itemize}

	\underline{ADVANCED STRESS PACKAGES}
	\begin{itemize}
	        \item The CONSTANT term used to report the rate of mass provided by a constant-concentration condition in the LKT, SFT, MWT, and UZT did not include the contribution to adjacent package features.  For example, if a stream reach was marked as constant-concentration boundary and it had flow into a downstream reach, then that flow was not included in the budget calculations.  Consequently, reported budgets in the listing file would show discrepancies that were larger than what was actually simulated by the model.  The program code was modified to include these mass flows to adjacent features.
	        \item The ET formulation in UZF was not reducing the residual pET passed to deeper UZF objects when the extinction depth spanned multiple UZF objects (layers).  As a result, too much water was removed when the water table was shallow.  Or, in some cases, water was removed from dry cells that were above the water table but within the extinction depth interval. The ET code within the UZF package was modified to remove only eligible water from the unsaturated and saturated zones.
            \item The UZF package should exit with an appropriate error message when SURFDEP > cell thickness.  When this condition is not enforced, bad mass balances may result.  
            \item The FLOW\_IMBALANCE\_CORRECTION implemented in the GWT FMI Package did not work properly with the GWF UZF Package.  The issue was fixed by ensuring that the FMI Package could accurately calculate the flow residual for cells that had a UZF entry.
            \item The SFR package should exit with an appropriate error message when a diversion has a cprior type of FRACTION but the divflow value is outside the range 0.0 to 1.0 as stated in the documentation.
	\end{itemize}

%	\underline{SOLUTION}
%	\begin{itemize}
%	        \item xxx  
%	        \item xxx  
%	        \item xxx  
%	\end{itemize}

\end{itemize}



\begin{itemize}

	\item Version mf6.2.0--October 20, 2020
	
	\underline{NEW FUNCTIONALITY}
	\begin{itemize}
	        \item A new Buoyancy (BUY) Package for the Groundwater Flow (GWF) Model is introduced in this release as a way to represent variable-density groundwater flow.  The BUY Package is based on the hydraulic head formulation described by \cite{langevin2020hydraulic}.  Extensive testing of the BUY Package has been performed but changes to the code and input may be required in response to user needs and testing.   
		\item A new Groundwater Transport (GWT) Model is introduced in this release as a way to simulate the fate and transport of a dissolved solute.  Extensive testing of the GWT Model has been performed but changes to the code and input may be required in response to user needs and testing. 
		\item The Basic Model Interface (BMI) capabilities were first released in version 6.1.1.  Extensive testing of the BMI capabilities has been performed but changes to the code and calling procedures be required in response to user needs and testing.
	\end{itemize}
	
	\underline{EXAMPLES}
	\begin{itemize}
	\item The format for the examples included in the distribution has changed.  The examples are now described in the modflow6-examples.pdf file in the doc folder.  The examples have been renamed, and they are no longer numbered.  Most of the examples are the same as those distributed with the previous release; however some have been modified, updated, combined or eliminated based on standardization of example construction, testing, and documentation.  
	\end{itemize}

	\textbf{\underline{BUG FIXES AND OTHER CHANGES TO EXISTING FUNCTIONALITY}} \\
	\underline{BASIC FUNCTIONALITY}
	\begin{itemize}
		\item The observation routines were improved to handle very large numbers of observations written to the same comma-separated-value (CSV) file.  Non-advancing input-output is now used to write the CSV header instead of constructing a header string.   This change should substantially improve memory and runtime problems with models containing thousands of observations.
		\item If the CONTINUE option is specified in mfsim.nam, do not force models to write budget tables when the solver does not converge.  Instead, always use Output Control options to determine when budget tables are written if CONTINUE option is specified and the solver does not converge.  Also, if the CONTINUE option is specified, calculate and write observations even if the model does not converge.  Observations were being written as zero if the model did not converge, but the CONTINUE flag was set.
		\item Allow the program to read input files with very long lines.  Previously, the program was limited to a maximum line length of 50,000 characters.  The program now uses dynamic memory allocation, when necessary, to read any sized line in an input file.
		\item Fixed an error in the implementation of the Newton-Raphson correction for XT3D. The error in the code would have only affected simulations that used the NEWTON option together with the XT3D RHS option. 
	\end{itemize}

	\underline{ADVANCED STRESS PACKAGES}
	\begin{itemize}
		\item Fix error in calculated Newton-Raphson MAW-GWF connection terms for the MAW Package. This correction should improve model convergence and may change existing model results. This correction does not affect simulations that use the FLOW\_CORRECTION option introduced in version 6.1.1.
		\item  The program will now terminate with an error message if the skin radius for a GWF connection in the MAW Package is less than or equal to the well radius. Warning messages are also issued when the well bottom, screen top for a GWF connection, or screen bottom for a GWF connection are reset by the program for one or more MAW Package wells.
		\item An SFR reach can have zero specified connections.  In this case, an entry is still required in the CONNECTIONDATA block for that reach.  The error check for this required entry was not implemented and so the program would continue with unexpected results.  The program now verifies that an entry is present in CONNECTIONDATA for every reach, even those with zero connections.
		\item If a GWF ``NONE'' connection was specified for SFR then the program would terminate with an error or proceed with unexpected results if GWF or SFR flow terms were written to binary output files.  The program was fixed so that GWF ``NONE'' connections for SFR are not written to the binary budget files.
		\item Increase the length of boundname to its intended size of 40 characters.  Boundnames were being truncated after 16 characters for the LAK, MAW, and SFR Packages.
	\end{itemize}

	\underline{SOLUTION}
	\begin{itemize}
		\item A fix was implemented in the biconjugate gradient stabilized linear solver routine so that the maximum change in the dependent value is calculated and stored correctly.
		\item Corrected the SIMPLE and COOLEY under-relaxation schemes in the Iterative Model Solution (IMS).  The methods were not applying the correct under-relaxation factor.  The SIMPLE scheme now uses the user-specified value for gamma as the factor.  The COOLEY scheme updates the factor based on solver history.  
	\end{itemize}

        

	\item Version mf6.1.1--June 12, 2020

	\underline{NEW FUNCTIONALITY}
	\begin{itemize}
		\item Refactor the source code to support the \href{https://csdms.colorado.edu/wiki/BMI_Description}{Basic Model Interface} (BMI) developed by the \href{https://csdms.colorado.edu/wiki/Main_Page}{Community Surface Dynamics Modeling System} (CSDMS) group. BMI is a set of standard control and query functions that, when added to a model code, make that model both easier to learn and easier to couple with other software elements \citep{PECKHAM20133}. Furthermore, the BMI makes it possible to control MODFLOW~6 execution from scripting languages using bindings for the BMI (for example, python bindings for the BMI available through \href{https://csdms.colorado.edu/wiki/PyMT}{pymt}). The BMI in this version is considered preliminary (alpha release). Limited testing of the BMI has been performed but significant changes are expected in future releases.  User support for the MODFLOW 6 BMI may be provided in the future.
		\item Add silent command line switch (\texttt{-s} or \texttt{\doubledash silent}) that sends all screen output (\texttt{STDOUT}) to a text file (with the name ``mfsim.stdout'').
		\item Add screen output command line switch (\texttt{-l <str>} or \texttt{\doubledash level <str>}) that controls output to the screen (\texttt{STDOUT}). If \texttt{<str>}  is \texttt{summary}, stress period and time step data are not written to \texttt{STDOUT}. If \texttt{<str>} is \texttt{debug}, normal and debug output are written to \texttt{STDOUT}. 
		\item Add simulation mode command line switch (\texttt{-m <str>} or \texttt{\doubledash mode <str>}) that controls the solution mode. If \texttt{<str>}  is \texttt{validate}, model input will be read and checked for errors, but the coefficient matrix or matrices will not be assembled or solved and solution output will not be written.
		\item Add SAVE\_SATURATION option to the Node Property Flow Package.  When invoked, cell saturation is written to the binary budget file as an auxiliary column for a record with the name ``DATA-SAT''.  The cell saturation can be used by post-processors to determine how much of the cell is saturated without having to know the value for ICELLTYPE or the value for head. If a cell is marked as confined (ICELLTYPE=0) then saturation is always one. If ICELLTYPE is one, then saturation ranges between zero and one.
		\item Add option for saving package convergence for the CSUB Package to a comma-separated values (CSV) file. Package convergence is enabled by specifying PACKAGE\_CONVERGENCE FILEOUT $<$package\_convergence\_filename$>$ in the options block for the package.
		\item Add option for saving package convergence for the LAK, SFR, and UZF Packages to comma-separated values (CSV) files. Package convergence for the LAK, SFR, and UZF Packages is enabled by specifying PACKAGE\_CONVERGENCE FILEOUT $<$package\_convergence\_filename$>$ in the options block for the package.
		\item Add CSV\_OUTER\_OUTPUT output option to save outer iteration information to a comma-separated values (CSV) file. The maximum of the model or package dependent-variable change for the outer iteration is written to the CSV file at the end of each outer iteration. 
		\item Add CSV\_INNER\_OUTPUT output option to save inner iteration information to a comma-separated values (CSV) file. The CSV output also the includes maximum dependent-variable change and maximum residual convergence information for the solution and each model (if the solution includes more than one model) and linear acceleration information for each inner iteration. The inner iteration CSV output, which contains a separate line for each inner iteration, is written to the CSV file all at once at the end of each outer iteration.
		\item Add OUTER\_DVCLOSE and INNER\_DVCLOSE variables to replace OUTER\_HCLOSE and INNER\_HCLOSE variables in the IMS Package input file. ``DV'' is used now instead to more generally refer to dependent variable.  Warning messages will be issued if OUTER\_HCLOSE and/or INNER\_HCLOSE variables are specified. OUTER\_HCLOSE and INNER\_HCLOSE variables will eventually be deprecated. 
		\item Add option to scale drain conductance as a function of simulated head over a user-defined range (drainage depth). Linear-conductance scaling is used with the Standard Formulation. Cubic-conductance scaling is used with the Newton-Raphson Formulation. The additional drainage depth variable is specified as an auxiliary variable and AUXDEPTHNAME is used to identify the auxiliary variable defining the drainage depth. The cubic-conductance scaling can be used as a replacement for the groundwater seepage option in the UZF Package. The scaled drainage conductance option can also be used to represent vertical seepage faces and improve model convergence in cells where simulated heads fluctuate around the elevation where the drain begins to discharge groundwater.
		\item Add timeseries support for the reach upstream fraction variable in the SFR package.
		\item Add Picard iterations for the SFR package to minimize differences in SFR package results between subsequent GWF Picard (non-linear) iterations as a result of non-optimal reach numbering. The number of SFR package Picard iterations can be controlled by specifying the maximum number of Picard  iteration to be used in the OPTIONS block (MAXIMUM\_PICARD\_ITERATIONS). If reaches are numbered in order, from upstream to downstream, MAXIMUM\_PICARD\_ITERATIONS can be set to 1 to reduce model run time. Specifying  MAXIMUM\_PICARD\_ITERATIONS to 1 will result in identical SFR package performance to previous versions of MODFLOW~6.
		\item Add flow correction option for the MAW package that corrects the MAW-GWF exchange in cases where the head in a multi-aquifer well is below the bottom of the screen for a connection or the head in a convertible cell connected to a multi-aquifer well is below the cell bottom.  When flow corrections are activated, unit head gradients are used to calculate the flow between a multi-aquifer well and a connected GWF cell. This option is identical to the MODFLOW-USG ``flow-to-dry-cell'' option for flow between a CLN cell and a GWF cell if the cell is convertible \citep{modflowusg}. Flow corrections are enabled by specifying FLOW\_CORRECTION in the OPTIONS block. By default, flow corrections are not made. \emph{Prior to this release (version 6.1.1), flow corrections were made anytime the head in a multi-aquifer well was below the bottom of the screen for a connection--this may result in different results for existing models that can be resolved by using the FLOW\_CORRECTION option.}
		\item Add new document, ``MODFLOW 6 -- Supplemental Technical Information,'' to the doc folder.  This document contains information that was in the mf6io.pdf appendices.  This technical information document may expand with future versions as new features are added.
	\end{itemize}

	\textbf{\underline{BUG FIXES AND OTHER CHANGES TO EXISTING FUNCTIONALITY}} \\
	\underline{BASIC FUNCTIONALITY}
	\begin{itemize}
		\item Correct an error in how the discretization package (for regular MODFLOW grids) calculates the distance between two cells when one or both of the cells are unconfined.  The error in the code would have only affected XT3D simulations with a regular grid, unconfined conditions, and specification of ANGLE2 in the NPF Package.  
		\item Correct an error in the use of the AUXMULTNAME option for boundary packages when time series are used.  A problem remains when time series are used for AUXMULTNAME but not for the column that is scaled by AUXMULTNAME.  This situation should be avoided.
	\end{itemize}

	\underline{STRESS PACKAGES}
	\begin{itemize}
		\item Fix a bug in binary budget file header for CSUB Package budget data written using IMETH=6 (CSUB-ELASTIC and CSUB-INELASTIC) .
		\item Add information on the CSUB Package budget terms and compaction data written the the Input/Output document in the `Description of Groundwater Flow (GWF) Model Binary Output Files' section.
		\item Prior to this release, calculated flows between a standard stress package (WEL, DRN, RIV, GHB, RCH, and EVT) and the connected model cell were based on the RHS and HCOF terms from the previous iteration.  This was not consistent with previous MODFLOW versions.  These packages were modified so that the flows are recalculated using the final converged head solution.  As a result of this change, simulated groundwater flows for these packages may be slightly different (compared to previous releases) if the package HCOF and RHS values depend on the simulated groundwater head.
	\end{itemize}

	\underline{ADVANCED STRESS PACKAGES}
	\begin{itemize}
		\item The code for saving the budget terms for the advanced packages was refactored to use common routines.  These changes should have no affect on simulation results.
		\item In previous releases, the LAK Package would accept negative user-input values for  RAINFALL, EVAPORATION, RUNOFF, INFLOW, and WITHDRAWAL even though the user guide mentioned that negative values are not allowed for these flow terms.  Error checks were added to ensure these values are specified as positive.
		\item Add a storage term to the SFR Package binary output file.  This term is always zero with the present implementation.  An auxiliary variable, called VOLUME, is also written with the storage budget term.  This term contains the calculated water volume in the reach.
		\item Refactor the SFR Package to remove use of RectangularChGeometry objects and added required functionality as private methods in the SFR module.
		\item  Improve error trapping in the MAW Package to catch divide by zero errors when calculating the saturated conductance for wells using the SKIN CONDEQN in connections where the cell  transmissivity (the product of geometric mean of the horizontal hydraulic conductivity and cell thickness) and well transmissivity (the product of HK\_SKIN and screen thickness) is equal to one. Also add error trapping for well connections using the 1) SKIN CONDEQN where the contrast between the cell and well transmissivities are less than one and 2) SKIN and MEAN CONDEQN where the calculated connection saturated conductance is less than zero.
		\item For the Lake Package, the outlet number was written as ID1 and ID2 for the TO-MVR record in the binary budget file.  This has been changed so that the lake number of the connected outlet is written to ID1 and ID2.  This change was implemented so that lake budgets can be calculated using the information in the lake budget file.
		\item The Lake, Streamflow Routing, and Multi-Aquifer Well Packages were modified to save the user-specified stage or head to the binary output file for lakes, reaches, or wells that are specified as being CONSTANT.  Prior to this change, a no-flow value was written to the package binary output files for constant stage lakes and streams and constant head multi-aquifer wells.  The no-flow value is still written for those lakes, streams, or wells that are specified by the user as being inactive.  This change should make it easier to post-process the results from these packages.
	\end{itemize}

	\underline{SOLUTION}
	\begin{itemize}
		\item Fix a bug in the linear solver when using the STRICT RCLOSE\_OPTION that prevented termination of inner iterations when the INNER\_DVCLOSE and INNER\_RCLOSE criteria were met but the inner iteration count was greater than one. The inner iterations are now terminated when the INNER\_DVCLOSE and INNER\_RCLOSE criteria are met but the linear solver is considered non-converged if the inner iteration count is greater than one.
		\item Deprecate the CSV\_OUTPUT output option in the OPTIONS BLOCK because the output to the comma-separated values (CSV) file was based on the PRINT\_OPTION option. If CSV\_OUTPUT is specified, it is used to define the file name for the CSV\_OUTER\_OUTPUT output option.
		\item Modify the outer iteration information written to the simulation listing file when PRINT\_OPTION is not NONE to improve the ability of users to evaluate model convergence. Added Package convergence data, eliminated dependent variable changes adjusted by under-relaxation, and flags to indicate when an outer iterations step is considered converged. Information is also provided if PTC causes non-convergence for a outer iteration (even if the model is converged) and if NEWTON UNDER\_RELAXATION resets outer iteration convergence from FALSE to TRUE. Dependent-variable changes for the under-relaxation step in an outer iteration are no longer reported because under-relaxation is only applied if the model or package outer iteration steps do not converge and by definition reduce dependent-variable changes and are not used to evaluate outer iteration convergence.
		\item Deprecate the OUTER\_RCLOSEBND optional variable in the NONLINEAR BLOCK because OUTER\_DVCLOSE is used for all terms used to evaluate package convergence. An warning will be issued if OUTER\_RCLOSEBND is specified.
		\item Deprecate the CSV\_OUTPUT output option. A warning will be issued if the CSV\_OUTPUT option is specified and outer iteration information will be saved to the specified FILEOUT comma-separated values (CSV) file.	
	\end{itemize}

\end{itemize}

\begin{itemize}
	\item Version mf6.1.0--December 12, 2019
	
	\underline{NEW FUNCTIONALITY}
	\begin{itemize}
		\item Added the Skeletal Storage, Compaction, and Subsidence (CSUB) Package. The one-dimensional effective-stress based compaction theory implemented in the CSUB Package is documented in \cite{leake2007modflow}. The numerical approach used for delay interbeds in the CSUB package is documented in \cite{hoffmann2003modflow} and uses the same one-dimensional effective-stress based compaction theory as coarse-grained and fine-grained no-delay interbed sediments. A number of example problems that use the CSUB Package are documented in the ``MODFLOW 6 CSUB Package Example Problems'' pdf document included in this and subsequent releases.
	\end{itemize}
	
	\underline{BASIC FUNCTIONALITY}
	\begin{itemize}
		\item Added an error check to the DISU Package that ensures that an underlying cell has a top elevation that is less than or equal to the bottom of an overlying cell.  An underlying cell is one in which the IHC value for the connection is zero and the connecting node number is greater than the cell node number.
		\item Added restricted IDOMAIN support for DISU grids.  Users can specify an optional IDOMAIN in the DISU Package input file.  IDOMAIN values must be zero or one.  Vertical pass-through cells (specified with an IDOMAIN value of -1 in the DIS or DISV Package input files) are not supported for DISU.   
		\item NPF Package will now write a message to the GWF Model list file to indicate when the SAVE\_SPECIFIC\_DISCHARGE option is invoked.
		\item Added two new options to the NPF Package.  The K22OVERK option allows the user to enter the anisotropy ratio for K22.  If activated, the K22 values entered by the user in the NPF input file will be multiplied by the K values entered in the NPF input file.  The K33OVERK option allows the user to enter the anisotropy ratio for K33.  If activated, the K33 values entered by the user in the NPF input file will be multiplied by the K values entered in the NPF input file.  With this K33OVERK option, for example, the user can specify a value of 0.1 for K33 and have all K33 values be one tenth of the values specified for K.  The program will terminate with an error if these options are invoked, but arrays for K22 and/or K33 are not provided in the NPF input file.
		\item Added new MAXERRORS option to mfsim.nam.  If specified, the maximum number of errors stored and printed will be limited to this number.  This can prevent a situation where memory will run out when there are an excessive number of errors.
		\item Refactored many parts of the code to remove unused variables, conform to stricter FORTRAN standard checks, and allow for new development efforts to be included in the code base.
	\end{itemize}
	
	\underline{STRESS PACKAGES}
	\begin{itemize}
		\item There was an error in the calculation of the segmented evapotranspiration rate for the case where the rate did not decrease with depth.  There was another error in which PETM0 was being used as the evapotranspiration rate at the surface instead of the proportion of the evapotranspiration rate at the surface.
	\end{itemize}
	
	\underline{ADVANCED STRESS PACKAGES}
	\begin{itemize}
		\item Corrected the way auxiliary variables are handled for the advanced packages.  In some cases, values for auxiliary variables were not being correctly written to the GWF Model budget file or to the advanced package budget file.  A consistent approach for updating and saving auxiliary variables was implemented for the MAW, SFR, LAK, and UZF Packages.
		\item The user guide was updated to include a missing laksetting that was omitted from the PERIOD block.  The laksetting description now includes an INFLOW option; a description for INFLOW is also now included.
		\item The LAK package was incorrectly making an error check against NOUTLETS instead of NLAKES.
		\item For the advanced stress packages, values assigned to the auxiliary variables were not written correctly to the GWF Model budget file, but the values were correct in the advanced package budget file.  Program was modified so that auxiliary variables are correctly written to the GWF Model budget file.
		\item Corrected several error messages issued by the SFR Package that were not formatted correctly.  
		\item Fixed a bug in which the lake stage stable would sometimes result in touching numbers.  This only occurred for negative lake stages.
		\item The UZF Package was built on the UZFKinematicType, which used an array of structures.  A large array like this, can cause memory problems.  The UZFKinematicType was replaced with a new UzfCellGroupType, which is a structure of arrays and is much more memory efficient.  The underlying UZF algorithm did not change.
	\end{itemize}
	
	\underline{SOLUTION}
	\begin{itemize}
		\item Add ALL and FIRST options to optional NO\_PTC optional keyword in OPTIONS block. If NO\_PTC option is FIRST, PTC is disabled for the first stress period but is applied in all subsequent steady-state stress periods. If NO\_PTC option is ALL, PTC is disabled for all steady-state stress periods. If the NO\_PTC options is not defined, PTC is disabled for all steady-state stress periods (this is consistent with the behaviour of the NO\_PTC option in previous versions).
	\end{itemize}
	
	\item Version mf6.0.4--Feb. 27, 2019
	
	\underline{BASIC FUNCTIONALITY}
	\begin{itemize}
		\item Addressed issue with pointing contiguous pointer vectors/arrays to non-contiguous pointer vectors/arrays that caused code compilation failure with gfortran-8. A consequence of addressing this issue is that all pointer vectors/arrays that are allocated or pointed to using the memory manager must be defined to be contiguous.
		\item Corrected a problem with the reading of grid data from a binary file, in which the program was reading a binary header for each row of data.
		\item Added a new error check for very small time steps.  If the value of the starting time is equal to the ending time (starting time plus the time step length), then the time step is too small to be differentiated by the program based on the precision of floating point numbers.  The program will terminate with an error in this case.  The program will also terminate if the storage package with a transient stress period has a time step length of zero.
		\item The observation package was modified to use non-advancing output instead of fixed length strings when writing ascii output. The previous use of fixed length strings resulted in truncation of ascii observation output when the product of user-specified \texttt{digits} + 7 and the number of observations exceeded 5000.
		\item Corrected an error in the GWF-GWF Exchange module that caused the specific discharge values in the child model to be calculated incorrectly.  The calculation was incorrect because the face normal for the child model was pointing toward the center of the cell instead of outward.
		\item Minor refactoring to improve code clarity.
	\end{itemize}
	
	\underline{STRESS PACKAGES}
	\begin{itemize}
		\item Minor refactoring to improve code clarity.
	\end{itemize}
	
	\underline{ADVANCED STRESS PACKAGES}
	\begin{itemize}
		\item Modified the Multi-Aquifer Well (MAW) Package so that the HEAD\_LIMIT and RATE\_SCALING options work for injection wells.  Prior to this change, these options only worked for extraction wells.  These options can be used to reduce or even shut off well injection as the head in the well rises above user-specified levels.
		\item Added stage and residual convergence checks to the SFR package to make sure that stage and upstream flow changes between successive outer iterations are less than OUTER\_HCLOSE and OUTER\_RCLOSEBND, respectively. This addition is expected to be useful for steady-state simulations with complicated networks and simple reaches.
		\item Modified the final convergence check for the LAK package to use OUTER\_HCLOSE when evaluating lake stage changes between successive outer iterations.
		\item Modified the final convergence check for the UZF package to use OUTER\_RCLOSEBND when evaluating rejected infiltration, groundwater recharge, and groundwater seepage changes between successive outer iterations.
		\item Minor refactoring to improve code clarity.
	\end{itemize}
	
	\underline{SOLUTION}
	\begin{itemize}
		\item Modified pseudo-transient continuation (PTC) approach to use PTC for steady-state stress period for models using the Newton-Raphson formulation for problems with and without the storage (STO) package. Previously, PTC was only used with problems that did not include the STO package (this was not the intended behavior of PTC).
		\item Added NO\_PTC option to disable PTC for problems where PTC degrades/prevents model convergence. Option only applies to steady-state stress periods for models using the Newton-Raphson formulation. For many problems, PTC can significantly improve convergence behavior for steady-state simulations, and for this reason it is active by default.  In some cases, however, PTC can worsen the convergence behavior, especially when the initial conditions are similar to the solution.  When the initial conditions are similar to, or exactly the same as, the solution and convergence is slow, then this NO\_PTC option should be used to deactivate PTC.  This NO\_PTC option should also be used in order to compare convergence behavior with other MODFLOW versions, as PTC is only available in MODFLOW~6. 
		\item Small improvements to PTC to reduce the initial PTCDEL value loaded on the diagonal. This reduces the number of iterations required to achieve convergence for steady-state stress periods for most problems.
		\item Added OUTER\_RCLOSEBND variable that is used when performing final convergence checks on model packages that solve a separate equation not solved by the IMS linear solver. This value represents the maximum allowable residual at any single model package element between successive outer iterations. An example of a model package that would use OUTER\_RCLOSEBND to evaluate convergence is the SFR package which solves a continuity equation for each reach.
		\item Minor refactoring to improve code clarity.
	\end{itemize}
	
	\item Version mf6.0.3--Aug. 9, 2018
	
	\underline{BASIC FUNCTIONALITY}
	\begin{itemize}
		\item Fixed issues with observations specified using boundnames that are enclosed in quotes. Previously, the closing quote was retained on a boundname enclosed in quotes and resulted in an error (the erroneous observation boundname could not be found in the package).
	\end{itemize}
	
	\underline{STRESS PACKAGES}
	\begin{itemize}
		\item If the AUXMULTNAME keyword was used in combination with time series, then the multiplier was erroneously applied to all time series, and not just the time series in the column to be scaled.  
		\item For the array-based recharge and evapotranspiration packages, the IRCH and IEVT variables (if specified) must be specified as the first variable listed in the PERIOD block.  A check was added so that the program will terminate with an error if IRCH or IEVT is not the first variable listed in the PERIOD block.
		\item For the standard boundary packages, the ``to mover'' term (such as DRN-TO-MVR) written to the GWF Model budget was incorrect.  The budget terms were incorrect because the accumulator variables were not initialized to zero. 
		\item For regular MODFLOW grids, the recharge and evapotranspiration arrays of size (NCOL, NROW) were being echoed to the listing file (if requested by the user) of size (NCOL * NROW). 
	\end{itemize}
	
	\underline{ADVANCED STRESS PACKAGES}
	\begin{itemize}
		\item Fixed spelling of the THIEM keyword in the source code and in the input instructions of the MAW Package.
		\item Fixed an issue with the SFR package when the specified evaporation exceeds the sum of specified and calculated reach inflows, rainfall, and specified runoff. In this case, evaporation is set equal to the sum of specified and calculated reach inflows, rainfall, and specified runoff. Also if a negative runoff is specified and this value exceeds specified and calculated reach inflows, and rainfall then runoff is set to the sum of reach inflows and evaporation is set to zero.
		\item Fixed an issue in the MAW package budget information written to the listing file and MAW cell-by-cell budget file when a previously active well is inactivated. The ratesim variable was not being reset to zero for these wells and the simulated rate from the last stress period when the well was active was being reported.
		\item Program now terminates with an error if the OUTLETS block is present in the LAK package file and NOUTLETS is not specified or specified to be zero in the DIMENSIONS block.  Previously, this did not cause an error condition in the LAK package but would result in a segmentation fault error in the MVR package if LAK package OUTLETS are specified as providers.
		\item Program now terminates with an error when a DIVERSION block is present in a SFR package file but no diversions (all ndiv values are 0) are specified in the PACKAGEDATA block. 
	\end{itemize}
	
	\underline{SOLUTION}
	\begin{itemize}
		\item Fixed bug related to not allocating the preconditioner work array if a non-zero drop tolerance is specified but the number of levels is not specified or specified to be zero. In the case where the number of levels is not specified or specified to be zero the preconditioner work array is dimensioned to the product of the number of cells (NEQ) and the maximum number of connections for any cell.
		\item Updated linear solver output so number of levels and drop tolerance are output if either are specified to be greater than zero. 
	\end{itemize}
	
	\item Version mf6.0.2--Feb 23, 2018
	
	\underline{BASIC FUNCTIONALITY}
	\begin{itemize}
		\item Added a new option, called SAVE\_SPECIFIC\_DISCHARGE to the Node Property Flow Package.  When invoked, $x$, $y$, and $z$ specific discharge components are calculated for the center of each model cell and written to the binary budget file.
		\item For binary input of grid data, such as initial heads, the array reading utility was not reading a header record consisting of KSTP, KPER, PERTIM, TOTIM, TEXT, NLAY, NROW, NCOL.  This meant that a binary head file written by MODFLOW could not be used as input for a subsequent simulation.  For binary input, the array reading utility now reads a header record before reading the array values.
		\item The NOGRB option in the discretization packages was not working.  This option will now prevent the binary grid file from being written.
		\item Removed the PRIVATE attribute for two methods of the discretization packages so that the program works as intended with the latest Intel Fortran release.
		\item Switched to using a long integer for the memory manager so that memory usage is calculated correctly for large models.
	\end{itemize}
	
	\underline{STRESS PACKAGES}
	\begin{itemize}
		\item If a steady-state stress period followed a transient stress period, the storage terms written to the budget file were not being reset to zero.  The program now initializes these budget values to zero for steady-state periods before they are written.
	\end{itemize}
	
	\underline{ADVANCED STRESS PACKAGES}
	\begin{itemize}
		\item The STATUS INACTIVE option was not working correctly for the MAW Package.
		\item Modified the MAW connection conductance calculation so that a linear relation between the water level in a cell and saturation is used for the standard formulation. In the previous version, the same quadratic saturation function was being used for the standard and Newton-Raphson formulation to calculate the MAW connection conductance. 
		\item Modified the MAW Package so that the top and bottom of the screen for a connection are reset to the top and bottom of the cell, respectively, for SPECIFIED, THEIM, SKIN, and CUMULATIVE conductance equations (CONDEQN). Also, the program will now terminate with an error if a MAW well using SPECIFIED, THEIM, SKIN, or CUMULATIVE conductance equations has more than one connection to a single GWF cell. 
		\item Modified the MAW package so that the well bottom (BOTTOM) is reset to the cell bottom in the lowermost GWF cell connection in cases where the specified well bottom is above the bottom of this GWF cell.
	\end{itemize}
	
	\underline{SOLUTION}
	\begin{itemize}
		\item Prior to applying pseudo transient continuation terms, the Iterative Model Solution confirms that the L2-norm exceeds the previous L2-norm.  If it doesn't then pseudo transient continuation is turned off.  This fixes a rare situation in which convergence could not be achieved for consecutive steady state solutions with the same or similar answers. 
	\end{itemize}
	
	
	\item Version mf6.0.1--Sep 28, 2017
	
	\underline{BASIC FUNCTIONALITY}
	\begin{itemize}
		\item There is no requirement that FTYPE entries in the GWF name file should be upper case; however, an upper case convention was being enforced.  FTYPE entries can now be specified using any case.
		\item Tab characters within model input files were not being skipped correctly.  This has been fixed.
		\item The program was updated to use the ``approved for release'' disclaimer.  The previous version was still using a ``preliminary software'' disclaimer.
		\item The source code for time series and time array series was refactored.  Included in the refactoring was a correction to time array series to allow the time array to change from one stress period to the next.  The source file TimeSeriesGroupList.f90 was renamed to TimeSeriesFileList.f90.
	\end{itemize}
	
	\underline{STRESS PACKAGES}
	\begin{itemize}
		\item Fixed inconsistency with CHD package observation name in code (\texttt{chd-flow}) and name in the input-output document (\texttt{chd}). Using name defined in input-output document (\texttt{chd}).
		\item The cell area was not being used in the calculation of recharge and evapotranspiration when list input was used with time series.
		\item The AUXMULTNAME option was not being applied for recharge and evapotranspiration when the READASARRAYS option was used.
		\item The program was not terminating with an error if a PERIOD block was encountered with an iper value equal to the previous iper value.  Program now terminates with an error.
	\end{itemize}
	
	\underline{ADVANCED STRESS PACKAGES}
	\begin{itemize}
		\item Fixed incorrect sign for SFR package exchange with GWF model (\texttt{sfr}).
		\item Added option to specify \texttt{none} as the \texttt{bedleak} for a lake-\texttt{GWF} connection in lake (LAK) package. This option makes the lake-\texttt{GWF} connection conductance solely a function of aquifer properties in the connected \texttt{GWF} cell and lakebed sediments are assumed to be absent for this connection.
		\item Fixed bug in lake (LAK) and multi-aquifer well (MAW) packages that only reset steady-state flag if lake and/or multi-aquifer data are read for a stress period (in the pak\_rp() routines). Using pointer to GWF iss variable in the LAK package and resetting the MAW steady state flag in maw\_rp() routine every stress period, regardless of whether MAW data are specified for a stress period.
		\item Added a convergence check routine to the GWF Mover Package that requires at least two outer iterations if there are any active movers.  Because mover rates are lagged by one outer iteration, at least two outer iterations are required for some problems.
		\item Changed the behavior of the LAK Package so that recharge and evapotranspiration are applied to a vertically connected GWF model cell if the lake status is INACTIVE.  Prior to this change, recharge and evapotranspiration were only applied to an underlying GWF model cell if the lake was dry.
	\end{itemize}
	
	\underline{SOLUTION}
	\begin{itemize}
		\item Fixed bug in IMS that allowed convergence when outer iteration HCLOSE value was satisfied but the model did not converge during the inner iterations.
		\item Added STRICT rclose\_option that uses a infinity-Norm RCLOSE criteria but requires HCLOSE and RCLOSE be satisfied on the first inner iteration of an outer iteration. The STRICT option is identical to the closure criteria approach use in the PCG Package in MODFLOW-2005.
	\end{itemize}
	
	\underline{EXCHANGES}
	\begin{itemize}
		\item Use of an OPEN/CLOSE file was not being allowed for the OPTIONS and DIMENSIONS blocks of the GWF6-GWF6 exchange input file.  OPEN/CLOSE input is now allowed for both of these blocks.
	\end{itemize}
	
	\item
	Version mf6.0.0---August 10, 2017
	
	\underline{BASIC FUNCTIONALITY}
	\begin{itemize}
		\item Removed support for the SINGLE observation type.  All observations must be CONTINUOUS, which means observation values are written for every time step. 
		\item Added support for a no-data value (3.0E30), which can be used as a placeholder in a time-series file containing multiple time series. Use of the no-data value facilitates combining separate time series into a single file when the time series contain records for differing simulation times.
		\item Model names specified in the simulation name file cannot have spaces in them.  A check was implemented to terminate with an error if the model name contains spaces.  Model names cannot exceed 16 characters.  Trailing spaces are allowed.
		\item The name and version of the compiler used to make the run file is now written to the terminal and to the simulation list file.
		\item Many of the Fortran source files were modified and reformatted.  Unused variables were removed.
	\end{itemize}
	
	\underline{ADVANCED STRESS PACKAGES}
	\begin{itemize}
		\item Updated MAW package so that well connection conductance calculations correctly account for THICKSTRT in the NPF package for layers that use THICKSTRT (and are confined).
		\item Added \texttt{CUMULATIVE} \texttt{coneqn} (conductance) option to MAW package.
		\item Fixed bug in LAK package weir lake outlet calculation.
		\item Fixed bug in LAK package when internal outlets were specified and combined with the MVR package that was also moving water internally in the same LAK package.
		\item Updated the table created when PRINT\_FLOWS is specified in the LAK package OPTIONS block to include internal flow terms if NOUTLETS is greater than 0. 
		\item Renamed Lake Tables DIMENSIONS block NENTRIES to NROW and added NCOL to DIMENSIONS block.
		\item Eliminated MAXIMUM\_OUTLET\_DEPTH = 10 [L] as default behavior for MANNING and WEIR LAK package lake outlet types. The maximum depth threshold was used in MODFLOW-2005 lake package because a table was used to calculate lake outflows to SFR. Can still use maximum depth threshold in develop versions of MODFLOW~6 by specifying MAXIMUM\_OUTLET\_DEPTH in the options block with a value.
		\item Removed MULTILAYER option for UZF package---this option didn't actually do anything.
		\item Added the requirement that the UZF number be specified as the first value on each line in the PACKAGEDATA block.
		\item Renamed MAXBOUND in the DIMENSIONS block of the SFR Package to be NREACHES.
		\item Implemented a check in the SFR Package to make sure that information is specified in the PACKAGEDATA block for every reach.  Program terminates with an error if information for a reach is not found.
	\end{itemize}
	
	\item
	Version mf6beta0.9.03---June 23, 2017
	
	\underline{BASIC FUNCTIONALITY}
	\begin{itemize}
		\item Renamed all FTYPE keywords to version 6.  They were named with an 8.  So, for example, the GHB Package is now activated in the GWF name file using ``GHB6'' instead of ``GHB8''.
		\item Keywords in the simulation name file must now be specified as TDIS6, GWF6, and GWF6-GWF6 to be consistent.
		\item The DIS Package had grid offsets (XOFFSET and YOFFSET) that could be specified as options.  These offsets were relative to the upper-left corner of the model grid.  The default value for YOFFSET was set to the sum of DELR so that (0, 0) would correspond to the lower-left corner of the model grid.  These options have been removed and replaced with XORIGIN and YORIGIN, which is the coordinate of the lower-left corner of the model grid.  The default value is zero for XORIGIN and YORIGIN.
		\item Can now specify XORIGIN, YORIGIN, and ANGROT as options for the DISV and DISU packages.  These values are written to the binary grid file, which can be used by post-processors to locate the model grid in space.  These options have no affect on the simulation results.  The default value is 0.0 if not specified.
		\item Added a new option to the TDIS input file called START\_DATE\_TIME.  This is a 30 character string that represents the simulation starting date and time, preferably in the format described at https://www.w3.org/TR/NOTE-datetime.  The value provided by the user has no affect on the simulation, but if it is provided, the value is written to the simulation list file.
		\item Changed default behavior for how memory usage is written to the end of the simulation list file.  Added new MEMORY\_PRINT\_OPTION to simulation options to control how memory usage is written.
		\item Corrections were made to the memory manager to ensure that all memory is deallocated at the end of a simulation.
	\end{itemize}
	
	\underline{INTERNAL FLOW PACKAGES}
	\begin{itemize}
		\item Changed the way hydraulic conductivity is specified in the NPF Package.  Users no longer specify HK, VK, and HANI.  Hydraulic conductivity is now specified as ``K''.  If hydraulic conductivity is isotropic, then this is all that needs to be specified.  For anisotropic cases, the user can specify an optional ``K22'' array and an optional ``K33'' array.  For an unrotated conductivity ellipsoid ``K22'' corresponds to hydraulic conductivity in the y direction and ``K33'' corresponds to hydraulic conductivity in the z direction, respectively.
	\end {itemize}
	
	\underline{ADVANCED STRESS PACKAGES}
	\begin{itemize}
		\item Modified the MAW Package to include the effects of aquifer anisotropy in the calculation of conductance.
		\item Simplified the SFR Package connectivity to reflect feedback from beta users. There is no longer a requirement to connect reaches that do not have flow between them.  Program will now terminate with an error if this condition is encountered.
		\item Added simple routing option to SFR package. This is the equivalent of the specified depth option (icalc=0) in previous versions of MODFLOW. If water is available in the reach, then there can be leakage from the SFR reach into the aquifer.  If no water is available, then no leakage is applied.  STAGE keyword also added and only applies to reaches that use the simple routing option. If the STAGE keyword is not specified for reaches that use the simple routing option the specified stage is set to the top of the reach (depth = 0).
		\item Added functionality to pass SFR leakage to the aquifer to the highest active layer.
		\item Converted SFR Manning's to a time-varying, time series aware variable.  
		\item Updated LAK package so that conductance calculations correctly account for THICKSTRT in the NPF package for layers that use THICKSTRT (and are confined). Also updated EMBEDDEDH and EMBEDDEDV so that the conductance for these connection types are constant for confined layers.
		\item Converted UZF stress period data to time series aware data.
		\item Added time-series aware AUXILIARY variables to UZF package.
		\item Implemented AUXMULTNAME in options block for UZF package (AUXILIARY variables have to be specified). AUXMULTNAME is applied to the GWF cell area and is used to simulated more than one UZF cell per GWF cell. This could be used to simulate different land use classifications (i.e., agricultural and natural land use types) in the same GWF cell.
	\end{itemize}
	
	\underline{SOLUTION}
	\begin{itemize}
		\item Reworked IMS convergence information so that model specific convergence information is also printed to each model listing file when PRINT\_OPTION ALL is specified in the IMS OPTIONS block.
		\item Added csv output option for IMS convergence information. Solution convergence information and model specific convergence information (if the solution includes more than one model) is written to a comma separated value file. If PRINT\_OPTION is NONE or SUMMARY, csv output includes maximum head change convergence information at the end of each outer iteration for each time step. If PRINT\_OPTION is ALL, csv output includes maximum head change and maximum residual convergence information for the solution and each model (if the solution includes more than one model) and linear acceleration information for each inner iteration. 
	\end{itemize}
	
	\item
	Version mf6beta0.9.02---May 19, 2017
	\begin{itemize}
		\item Renamed gwf3.f90 to be lower case.
		\item Added the missing ``divrate'' variable to the ``sfrsetting'' description in mf6io.pdf.
		\item Added additional error trapping to the array reading utilities.
		\item There was a problem with the binary budget file when a GWF Exchange was used to connect a GWF Model with itself.  This has been fixed.
		\item Standardized `\texttt{to-mvr}' cell-by-cell item in standard stress packages and UZF package.
		\item Fixed incorrect `\texttt{UZF-EVT}' budget accumulator used in GWF listing budget. 
		\item Standardized justification of cell-by-cell `\texttt{text}' strings.
		\item Standardized use of AUXILIARY keyword.
	\end{itemize}
	
	\item
	Version mf6beta0.9.01---May 11, 2017
	\begin{itemize}
		\item Added a copy of the third MODFLOW~6 report. 
		\item Made several minor corrections to doc/mf6io.pdf.  
		\item If vertices were specified for DISU, then the last header line was not written to the binary grid file.  This has been corrected.
	\end{itemize}
	
	\item
	Version mf6beta0.9.00---May 10, 2017
	\begin{itemize}
		\item First public release of MODFLOW~6 in beta form. 
	\end{itemize}

\end{itemize}