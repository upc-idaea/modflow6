This section describes the data files for a \mf Groundwater Flow (GWF) Model.  A GWF Model is added to the simulation by including a GWF entry in the MODELS block of the simulation name file.

There are three types of spatial discretization approaches that can be used with the GWF Model.  Input for a GWF Model may be entered in a structured form, like for previous MODFLOW versions, in that users specify cells using their layer, row, and column indices.  Users may also work with a layered grid in which cells are defined using vertices.  In this case, users specify cells using the layer number and the cell number.  Lastly, GWF Models may be entered as fully unstructured models, in which cells are specified using only their cell number.  Once a spatial discretization approach has been selected, then all input with cell indices must be entered accordingly.

The GWF Model is designed to permit input to be gathered, as it is needed, from many different files.  Likewise, results from the model calculations can be written to a number of output files. The GWF Model Listing File is a key file to which the GWF model output is written.  As \mf runs, information about the GWF Model is written to the GWF Model Listing File, including much of the input data (as a record of the simulation) and calculated results.  Details about the files used by each package are provided in this section on the GWF Model Instructions.

\mf is further designed to allow the user to control the amount, type, and frequency of information to be output. Much of the output will be written to the Simulation and GWF Model Listing Files, but some model output can be written to other files.  The Listing Files can become very large for common models.  Text editors are useful for examining the Listing File. The GWF Model Listing File includes a summary of the input data read for all packages.  In addition, the GWF Model Listing File optionally contains calculated head controlled by time step, and the overall volumetric budget controlled by time step. The Listing Files also contain information about solver convergence and error messages.  Output to other files can include head and cell-by-cell flow terms for use in calculations external to the model or in user-supplied applications such as plotting programs.

The GWF Model reads a file called the Name File, which specifies most of the files that will be used in a simulation. Several files are always required whereas other files are optional depending on the simulation. The Output Control Package receives instructions from the user to control the amount and frequency of output.  Details about the Name File and the Output Control Package are described in this section.

\subsection{Information for Existing MODFLOW Users}
\input{gwf/info_existing_users.tex}

\input{gwf/array_data.tex}

\subsection{Units of Length and Time}
The GWF Model formulates the groundwater flow equation without using prescribed length and time units. Any consistent units of length and time can be used when specifying the input data for a simulation. This capability gives a certain amount of freedom to the user, but care must be exercised to avoid mixing units.  The program cannot detect the use of inconsistent units.  For example, if hydraulic conductivity is entered in units of feet per day and pumpage as cubic meters per second, the program will run, but the results will be meaningless. Other processes generally are expected to work with consistent length and time units; however, other processes could conceivably place restrictions on which units are supported.

The user can set flags that specify the length and time units (see the input instructions for the Timing Module and Spatial Discretization Files), which may be useful in various parts of MODFLOW.  For example, the program will label the table of simulation time with time units if the time units are specified by the optional TIME\_UNITS label, which can be set in the TDIS Package.  If the time units are not specified, the program still runs, but the table of simulation time does not indicate the time units. An optional LENGTH\_UNITS label can be set in the Discretization Package. Situations in other processes may require that the length or time units be specified.  In such situations, the input instructions will state the requirements. Remember that specifying the unit flags does not enforce consistent use of units.  The user must insure that consistent units are used in all input data.

\subsection{Steady-State Simulations}
A steady-state simulation is represented by a single stress period having a single time step with the storage term set to zero. Setting the number and length of stress periods and time steps is the responsibility of the Timing Module of the \mf framework. The length of the stress period and time step will not affect the head solution because the time derivative is not calculated in a steady-state problem. Setting the storage term to zero is the responsibility of the Storage Package. Most other packages need not "know" that a simulation is steady state.

A GWF Model also can be mixed transient and steady state because each stress period can be designated transient or steady state.  Thus, a GWF Model can start with a steady-state stress period and continue with one or more transient stress periods.  The settings for controlling steady-state and transient options are in the Storage Package.  If the Storage Package is not specified for a GWF Model, then the storage terms are zero and the GWF Model will be steady state.

\subsection{Volumetric Budget}
A summary of all inflows (sources) and outflows (sinks) of water is called a water budget.  The water budget for the GWF Model is termed a volumetric budget because volumes of water and volumetric flow rates are involved; thus strictly speaking, a volumetric budget is not a mass balance, although this term has been used in other model reports.  \mf calculates a water budget for the overall model as a check on the acceptability of the solution, and to provide a summary of the sources and sinks of water to the flow system.  The water budget is printed to the GWF Model Listing File for selected time steps.

Numerical solution techniques for simultaneous equations do not always result in a correct answer; in particular, iterative solvers may stop iterating before a sufficiently close approximation to the solution is attained.  A water budget provides an indication of the overall acceptability of the solution.  The system of equations solved by the model actually consists of a flow continuity statement for each model cell.  Continuity should also exist for the total flows into and out of the model---that is, the difference between total inflow and total outflow should equal the total change in storage.  In the model program, the water budget is calculated independently of the equation solution process, and in this sense may provide independent evidence of a valid solution.

The total budget as printed in the output does not include internal flows between model cells---only flows into or out of the model as a whole. For example, flow to or from rivers, flow to or from constant-head cells, and flow to or from wells are all included in the overall budget terms.  Flow into and out of storage is also considered part of the overall budget inasmuch as accumulation in storage effectively removes water from the flow system and storage release effectively adds water to the flow---even though neither process, in itself, involves the transfer of water into or out of the ground-water regime.  Each hydrologic package calculates its own contribution to the budget.

For every time step, the budget subroutine of each hydrologic package calculates the rate of flow into and out of the system due to the process simulated by the package.  The inflows and outflows for each component of flow are stored separately.  Most packages deal with only one such component of flow.  In addition to flow, the volumes of water entering and leaving the model during the time step are calculated as the product of flow rate and time-step length.  Cumulative volumes, from the beginning of the simulation, are then calculated and stored.

The GWF Model uses the inflows, outflows, and cumulative volumes to write the budget to the Listing File at the times requested by the model user.  When a budget is written, the flow rates for the last time step and cumulative volumes from the beginning of simulation are written for each component of flow.  Inflows are written separately from outflows.  Following the convention indicated above, water entering storage is treated as an outflow (that is, as a loss of water from the flow system) while water released from storage is treated as an inflow (that is, a source of water to the flow system).  In addition, total inflow and total outflow are written, as well as the difference between total inflow and outflow.  The difference is then written as a percentage error, calculated using the formula:

\begin{equation}
D = \frac{100 (IN-OUT)}{(IN + OUT) / 2}
\end{equation}

\noindent where $D$ is the percentage error term, $IN$ is the total inflow to the system, and $OUT$ is the total outflow.

If the model equations are solved correctly, the percentage error should be small.  In general, flow rates may be taken as an indication of solution validity for the time step to which they apply, while cumulative volumes are an indication of validity for the entire simulation up to the time of the output.  The budget is written to the GWF Model Listing File at the end of each stress period whether requested or not.

\subsection{Cell-By-Cell Flows}
In some situations, calculating flow terms for various subregions of the model is useful.  To facilitate such calculations, provision has been made to save flow terms for individual cells in a separate binary file so they can be used in computations external to the model itself.  These individual cell flows are referred to here as ``cell-by-cell'' flow terms and are of four general types: (1) cell-by-cell stress flows, or flows into or from an individual cell caused by one of the external stresses represented in the model, such as evapotranspiration or recharge; (2) cell-by-cell storage terms, which give the rate of accumulation or depletion of storage in an individual cell; and (3) internal cell-by-cell flows, which are actually the flows across individual cell faces---that is, between adjacent model cells.  These four kinds of cell-by-cell flow terms are discussed further in subsequent paragraphs.  To save any of these cell-by-cell terms, two flags in the model input must be set.  The input to the Output Control file indicates the time steps for which cell-by-cell terms are to be saved. In addition, each hydrologic package includes an option called SAVE\_FLOWS that must be set if the cell-by-cell terms computed by that package are to be saved.  Thus, if the appropriate option in the Evapotranspiration Package input is set, cell-by-cell evapotranspiration terms will be saved for each time step for which the saving of cell-by-cell flow is requested through the Output Control Option.  Only flow values are saved in the cell-by-cell files; neither water volumes nor cumulative water volumes are included.  The flow dimensions are volume per unit time, where volume and time are in the same units used for all model input data.  The cell-by-cell flow values are stored in unformatted form to make the most efficient use of disk space; see the Budget File section toward the end of this user guide for information on how the data are written to a file.

The cell-by-cell storage term gives the net flow to or from storage in a variable-head cell.  The net storage for each cell in the grid is saved in transient simulations if the appropriate flags are set.  Withdrawal from storage in the cell is considered positive, whereas accumulation in storage is considered negative.

The cell-by-cell constant-head flow term gives the flow into or out of an individual constant-head cell (specified with the CHD Package).  This term is always associated with the constant-head cell itself, rather than with the surrounding cells that contribute or receive the flow.  A constant-head cell may be surrounded by as many as six adjacent variable-head cells for a regular grid or any number of cells for the other grid types.  The cell-by-cell calculation provides a single flow value for each constant-head cell, representing the algebraic sum of the flows between that cell and all of the adjacent variable-head cells.  A positive value indicates that the net flow is away from the constant-head cell (into the variable-head part of the grid); a negative value indicates that the net flow is into the constant-head cell.

The internal cell-by-cell flow values represent flows across the individual faces of a model cell.  Flows between cells are written in the compressed row storage format, whereby the flow between cell $n$ and each one of its connecting $m$ neighbor cells are contained in a single one-dimensional array.  Flows are positive for the cell in question.  Thus the flow reported for cell $n$ and its connection with cell $m$ is opposite in sign to the flow reported for cell $m$ and its connection with cell $n$.  These internal cell-by-cell flow values are useful in calculations of the groundwater flow into various subregions of the model, or in constructing flow vectors.

Cell-by-cell stress flows are flow rates into or out of the model, at a particular cell, owing to one particular external stress.  For example, the cell-by-cell evapotranspiration term for cell $n$ would give the flow out of the model by evapotranspiration from cell $n$.  Cell-by-cell stress flows are considered positive if flow is into the cell, and negative if out of the cell.

\newpage
\subsection{GWF Model Name File}
\input{gwf/namefile.tex}

\newpage
\subsection{Structured Discretization (DIS) Input File}
\input{gwf/dis}

\newpage
\subsection{Discretization by Vertices (DISV) Input File}
\input{gwf/disv}

\newpage
\subsection{Unstructured Discretization (DISU) Input File}
\input{gwf/disu}

\newpage
\subsection{Initial Conditions (IC) Package}
\input{gwf/ic}

\newpage
\subsection{Output Control (OC) Option}
\input{gwf/oc}

\newpage
\subsection{Observation (OBS) Utility for a GWF Model}
\input{gwf/gwf-obs}

\newpage
\subsection{Node Property Flow (NPF) Package}
\input{gwf/npf}

\newpage
\subsection{Time-Varying Hydraulic Conductivity (TVK) Package}
Input to the Time-Varying Hydraulic Conductivity (TVK) Package is read from the file that is specified in the TVK6 record of the OPTIONS block in the NPF package.

\vspace{5mm}
\subsubsection{Structure of Blocks}
\vspace{5mm}

\noindent \textit{FOR EACH SIMULATION}
\lstinputlisting[style=blockdefinition]{./mf6ivar/tex/utl-tvk-options.dat}
\vspace{5mm}
\noindent \textit{FOR ANY STRESS PERIOD}
\lstinputlisting[style=blockdefinition]{./mf6ivar/tex/utl-tvk-period.dat}

\vspace{5mm}
\subsubsection{Explanation of Variables}
\begin{description}
% DO NOT MODIFY THIS FILE DIRECTLY.  IT IS CREATED BY mf6ivar.py 

\item \textbf{Block: OPTIONS}

\begin{description}
\item \texttt{TS6}---keyword to specify that record corresponds to a time-series file.

\item \texttt{FILEIN}---keyword to specify that an input filename is expected next.

\item \texttt{ts6\_filename}---defines a time-series file defining time series that can be used to assign time-varying values. See the ``Time-Variable Input'' section for instructions on using the time-series capability.

\end{description}
\item \textbf{Block: PERIOD}

\begin{description}
\item \texttt{iper}---integer value specifying the starting stress period number for which the data specified in the PERIOD block apply.  IPER must be less than or equal to NPER in the TDIS Package and greater than zero.  The IPER value assigned to a stress period block must be greater than the IPER value assigned for the previous PERIOD block.  The information specified in the PERIOD block will continue to apply for all subsequent stress periods, unless the program encounters another PERIOD block.

\item \texttt{cellid}---is the cell identifier, and depends on the type of grid that is used for the simulation.  For a structured grid that uses the DIS input file, CELLID is the layer, row, and column.   For a grid that uses the DISV input file, CELLID is the layer and CELL2D number.  If the model uses the unstructured discretization (DISU) input file, CELLID is the node number for the cell.

\item \texttt{tvksetting}---line of information that is parsed into a property name keyword and values.  Property name keywords that can be used to start the TVKSETTING string include: K, K22, and K33.

\begin{lstlisting}[style=blockdefinition]
K <@k@>
K22 <@k22@>
K33 <@k33@>
\end{lstlisting}

\item \textcolor{blue}{\texttt{k}---is the new value to be assigned as the cell's hydraulic conductivity from the start of the specified stress period, as per K in the NPF package.  If the OPTIONS block includes a TS6 entry (see the ``Time-Variable Input'' section), values can be obtained from a time series by entering the time-series name in place of a numeric value.}

\item \textcolor{blue}{\texttt{k22}---is the new value to be assigned as the cell's hydraulic conductivity of the second ellipsoid axis (or the ratio of K22/K if the K22OVERK NPF package option is specified) from the start of the specified stress period, as per K22 in the NPF package.  For an unrotated case this is the hydraulic conductivity in the y direction.  If the OPTIONS block includes a TS6 entry (see the ``Time-Variable Input'' section), values can be obtained from a time series by entering the time-series name in place of a numeric value.}

\item \textcolor{blue}{\texttt{k33}---is the new value to be assigned as the cell's hydraulic conductivity of the third ellipsoid axis (or the ratio of K33/K if the K33OVERK NPF package option is specified) from the start of the specified stress period, as per K33 in the NPF package.  For an unrotated case, this is the vertical hydraulic conductivity.  If the OPTIONS block includes a TS6 entry (see the ``Time-Variable Input'' section), values can be obtained from a time series by entering the time-series name in place of a numeric value.}

\end{description}


\end{description}

\vspace{5mm}
\subsubsection{Example Input File}
\lstinputlisting[style=inputfile]{./mf6ivar/examples/utl-tvk-example.dat}


\newpage
\subsection{Horizontal Flow Barrier (HFB) Package}
\input{gwf/hfb}

\newpage
\subsection{Storage (STO) Package}
\input{gwf/sto}

\newpage
\subsection{Time-Varying Storage (TVS) Package}
Input to the Time-Varying Storage (TVS) Package is read from the file that is specified in the TVS6 record of the OPTIONS block in the STO package.

\vspace{5mm}
\subsubsection{Structure of Blocks}
\vspace{5mm}

\noindent \textit{FOR EACH SIMULATION}
\lstinputlisting[style=blockdefinition]{./mf6ivar/tex/utl-tvs-options.dat}
\vspace{5mm}
\noindent \textit{FOR ANY STRESS PERIOD}
\lstinputlisting[style=blockdefinition]{./mf6ivar/tex/utl-tvs-period.dat}

\vspace{5mm}
\subsubsection{Explanation of Variables}
\begin{description}
% DO NOT MODIFY THIS FILE DIRECTLY.  IT IS CREATED BY mf6ivar.py 

\item \textbf{Block: OPTIONS}

\begin{description}
\item \texttt{DISABLE\_STORAGE\_CHANGE\_INTEGRATION}---keyword that deactivates inclusion of storage derivative terms in the STO package matrix formulation.  In the absence of this keyword (the default), the groundwater storage formulation will be modified to correctly adjust heads based on transient variations in stored water volumes arising from changes to SS and SY properties.

\item \texttt{TS6}---keyword to specify that record corresponds to a time-series file.

\item \texttt{FILEIN}---keyword to specify that an input filename is expected next.

\item \texttt{ts6\_filename}---defines a time-series file defining time series that can be used to assign time-varying values. See the ``Time-Variable Input'' section for instructions on using the time-series capability.

\end{description}
\item \textbf{Block: PERIOD}

\begin{description}
\item \texttt{iper}---integer value specifying the starting stress period number for which the data specified in the PERIOD block apply.  IPER must be less than or equal to NPER in the TDIS Package and greater than zero.  The IPER value assigned to a stress period block must be greater than the IPER value assigned for the previous PERIOD block.  The information specified in the PERIOD block will continue to apply for all subsequent stress periods, unless the program encounters another PERIOD block.

\item \texttt{cellid}---is the cell identifier, and depends on the type of grid that is used for the simulation.  For a structured grid that uses the DIS input file, CELLID is the layer, row, and column.   For a grid that uses the DISV input file, CELLID is the layer and CELL2D number.  If the model uses the unstructured discretization (DISU) input file, CELLID is the node number for the cell.

\item \texttt{tvssetting}---line of information that is parsed into a property name keyword and values.  Property name keywords that can be used to start the TVSSETTING string include: SS and SY.

\begin{lstlisting}[style=blockdefinition]
SS <@ss@>
SY <@sy@>
\end{lstlisting}

\item \textcolor{blue}{\texttt{ss}---is the new value to be assigned as the cell's specific storage (or storage coefficient if the STORAGECOEFFICIENT STO package option is specified) from the start of the specified stress period, as per SS in the STO package.  Specific storage values must be greater than or equal to 0.  If the OPTIONS block includes a TS6 entry (see the ``Time-Variable Input'' section), values can be obtained from a time series by entering the time-series name in place of a numeric value.}

\item \textcolor{blue}{\texttt{sy}---is the new value to be assigned as the cell's specific yield from the start of the specified stress period, as per SY in the STO package.  Specific yield values must be greater than or equal to 0.  If the OPTIONS block includes a TS6 entry (see the ``Time-Variable Input'' section), values can be obtained from a time series by entering the time-series name in place of a numeric value.}

\end{description}


\end{description}

\vspace{5mm}
\subsubsection{Example Input File}
\lstinputlisting[style=inputfile]{./mf6ivar/examples/utl-tvs-example.dat}


\newpage
\subsection{Skeletal Storage, Compaction, and Subsidence (CSUB) Package}
\input{gwf/csub}

\newpage
\subsection{Buoyancy (BUY) Package}
Input to the Buoyancy (BUY) Package is read from the file that has type ``BUY6'' in the Name File.  If the BUY Package is included for a model, then the model will use the variable-density form of Darcy's Law for all flow calculations using the approach described by \cite{langevin2020hydraulic}.  Only one BUY Package can be specified for a GWF model. The BUY Package can be coupled with one or more GWT Models so that fluid density is updated dynamically with one or more simulated concentration fields.

The BUY Package calculates fluid density using the following equation of state from \cite{langevin2008seawat}:

\begin{equation}
\label{eqn:volumeconservationdiscrete}
%\rho = \rho_0 + \sum_{i=1}^{NRHOSPECIES} \left ( C_i - C_{i,0} \right )
\rho = DENSEREF + \sum_{i=1}^{NRHOSPECIES} DRHODC_i \left ( CONCENTRATION_i - CRHOREF_i \right )
\end{equation}

\noindent where $\rho$ is the calculated density, $DENSEREF$ is the density of a reference fluid, typically taken to be freshwater at a temperature of 25 degrees Celsius; $NRHOSPECIES$ is the number of chemical species that contribute to the density calculation, $DRHODC_i$ is the parameter that describes how density changes as a function of concentration for chemical species $i$ (i.e. the slope of a line that relates density to concentration), $CONCENTRATION_i$ is the concentration of species $i$, and $CRHOREF_i$ is the concentration of species $i$ in the reference fluid, which is normally set to zero.

\subsubsection{Stress Packages}
For head-dependent stress packages, the BUY Package may require fluid density and elevation for each head-dependent boundary so that the model can use a variable-density form of Darcy's Law to calculate flow between the boundary and the aquifer.  By default, the boundary elevation is set equal to the cell elevation.  For water-table conditions, the cell elevation is calculated as bottom elevation plus half of saturation multiplied by the cell thickness.  If desired, the user can more precisely locate the boundary elevation by specifying an auxiliary variable with the name ``ELEVATION''.  The program will use the values in this column as the boundary elevation.  A situation where this may be required is for river or general-head boundaries that are conceptualized as being on top of a model cell.  In those cases, an ELEVATION column should be specified and the values set to the top of the cell or some other appropriate elevation that corresponds to where the boundary stage applies.

By default, the boundary density is set equal to DENSEREF, commonly specified as the density of freshwater; however, there are two other options for setting the density of a boundary package.  The first is to assign an auxiliary variable with the name ``DENSITY''.  If this auxiliary variable is detected, then the density value in this column will be assigned to the density for the boundary.  Alternatively, a density value can be calculated for each boundary using the density equation of state and one or more concentrations provided as auxiliary variables.  In this case, the user must assign one auxiliary variable for each AUXSPECIESNAME listed in the PACKAGEDATA block below.  Thus, there must be NRHOSPECIES auxiliary variables, each with the identical name as those specified in PACKAGEDATA.  The BUY Package will calculate the density for each boundary using these concentrations and the values specified for DENSEREF, DRHODC, and CRHOREF.  If the boundary package contains an auxiliary variable named DENSITY and also contains AUXSPECIESNAME auxiliary variables, then the boundary density value will be assigned to the one in the DENSITY auxiliary variable.

A GWT Model can be used to calculate concentrations for the advanced stress packages (LAK, SFR, MAW, and UZF) if corresponding advanced transport packages are specified (LKT, SFT, MWT, and UZT).  The advanced stress packages have an input option called FLOW\_PACKAGE\_AUXILIARY\_NAME.  When activated, this option will result in the simulated concentration for a lake or other feature being copied from the advanced transport package into the auxiliary variable for the corresponding GWF stress package.  This means that the density for a lake or stream, for example, can be dynamically updated during the simulation using concentrations from advanced transport packages that are fed into auxiliary variables in the advanced stress packages, and ultimately used by the BUY Package to calculate a fluid density using the equation of state.  This concept also applies when multiple GWT Models are used simultaneously to simulate multiple species.  In this case, multiple auxiliary variables are required for an advanced stress package, with each one representing a concentration from a different GWT Model.  

\begin{longtable}{p{3cm} p{12cm}}
\caption{Description of density terms for stress packages}
\tabularnewline
\hline
\hline
\textbf{Stress Package} & \textbf{Note} \\
\hline
\endhead
\hline
\endfoot
GHB & ELEVATION can be specified as an auxiliary variable.  A DENSITY auxiliary variable or one or more auxiliary variables for calculating density in the equation of state can be specified \\
RIV & ELEVATION can be specified as an auxiliary variable.  A DENSITY auxiliary variable or one or more auxiliary variables for calculating density in the equation of state can be specified \\
DRN & The drain formulation assumes that the drain boundary contains water of the same density as the discharging water; auxiliary variables have no effect on the drain calculation  \\
LAK & Elevation for each lake-aquifer connection is determined based on lake bottom and adjacent cell elevations. A DENSITY auxiliary variable or one or more auxiliary variables for calculating density in the equation of state can be specified \\
SFR & Elevation for each sfr-aquifer connection is determined based on stream bottom and adjacent cell elevations. A DENSITY auxiliary variable or one or more auxiliary variables for calculating density in the equation of state can be specified \\
MAW & Elevation for each maw-aquifer connection is determined based on cell elevation. A DENSITY auxiliary variable or one or more auxiliary variables for calculating density in the equation of state can be specified \\
UZF & No density terms implemented \\
\end{longtable}

\vspace{5mm}
\subsubsection{Structure of Blocks}

\vspace{5mm}
\noindent \textit{FOR EACH SIMULATION}
\lstinputlisting[style=blockdefinition]{./mf6ivar/tex/gwf-buy-options.dat}
\lstinputlisting[style=blockdefinition]{./mf6ivar/tex/gwf-buy-dimensions.dat}
\lstinputlisting[style=blockdefinition]{./mf6ivar/tex/gwf-buy-packagedata.dat}
%\vspace{5mm}
%\noindent \textit{FOR ANY STRESS PERIOD}
%\lstinputlisting[style=blockdefinition]{./mf6ivar/tex/gwf-buy-period.dat}

\vspace{5mm}
\subsubsection{Explanation of Variables}
\begin{description}
% DO NOT MODIFY THIS FILE DIRECTLY.  IT IS CREATED BY mf6ivar.py 

\item \textbf{Block: OPTIONS}

\begin{description}
\item \texttt{HHFORMULATION\_RHS}---use the variable-density hydraulic head formulation and add off-diagonal terms to the right-hand.  This option will prevent the BUY Package from adding asymmetric terms to the flow matrix.

\item \texttt{denseref}---fluid reference density used in the equation of state.  This value is set to 1000. if not specified as an option.

\item \texttt{DENSITY}---keyword to specify that record corresponds to density.

\item \texttt{FILEOUT}---keyword to specify that an output filename is expected next.

\item \texttt{densityfile}---name of the binary output file to write density information.  The density file has the same format as the head file.  Density values will be written to the density file whenever heads are written to the binary head file.  The settings for controlling head output are contained in the Output Control option.

\end{description}
\item \textbf{Block: DIMENSIONS}

\begin{description}
\item \texttt{nrhospecies}---number of species used in density equation of state.  This value must be one or greater.  The value must be one if concentrations are specified using the CONCENTRATION keyword in the PERIOD block below.

\end{description}
\item \textbf{Block: PACKAGEDATA}

\begin{description}
\item \texttt{irhospec}---integer value that defines the species number associated with the specified PACKAGEDATA data on the line. IRHOSPECIES must be greater than zero and less than or equal to NRHOSPECIES. Information must be specified for each of the NRHOSPECIES species or the program will terminate with an error.  The program will also terminate with an error if information for a species is specified more than once.

\item \texttt{drhodc}---real value that defines the slope of the density-concentration line for this species used in the density equation of state.

\item \texttt{crhoref}---real value that defines the reference concentration value used for this species in the density equation of state.

\item \texttt{modelname}---name of GWT model used to simulate a species that will be used in the density equation of state.  This name will have no effect if the simulation does not include a GWT model that corresponds to this GWF model.

\item \texttt{auxspeciesname}---name of an auxiliary variable in a GWF stress package that will be used for this species to calculate a density value.  If a density value is needed by the Buoyancy Package then it will use the concentration values in this AUXSPECIESNAME column in the density equation of state.  For advanced stress packages (LAK, SFR, MAW, and UZF) that have an associated advanced transport package (LKT, SFT, MWT, and UZT), the FLOW\_PACKAGE\_AUXILIARY\_NAME option in the advanced transport package can be used to transfer simulated concentrations into the flow package auxiliary variable.  In this manner, the Buoyancy Package can calculate density values for lakes, streams, multi-aquifer wells, and unsaturated zone flow cells using simulated concentrations.

\end{description}


\end{description}

\vspace{5mm}
\subsubsection{Example Input File}
\lstinputlisting[style=inputfile]{./mf6ivar/examples/gwf-buy-example.dat}



\newpage
\subsection{Constant-Head (CHD) Package}
\input{gwf/chd}

\newpage
\subsection{Well (WEL) Package}
\input{gwf/wel}

\newpage
\subsection{Drain (DRN) Package}
\input{gwf/drn}

\newpage
\subsection{River (RIV) Package}
\input{gwf/riv}

\newpage
\subsection{General-Head Boundary (GHB) Package}
\input{gwf/ghb}

\newpage
\subsection{Recharge (RCH) Package -- List-Based Input}
\input{gwf/rch}

\newpage
\subsection{Recharge (RCH) Package -- Array-Based Input}
\input{gwf/rcha}

\newpage
\subsection{Evapotranspiration (EVT) Package -- List-Based Input}
\input{gwf/evt}

\newpage
\subsection{Evapotranspiration (EVT) Package -- Array-Based Input}
\input{gwf/evta}

\newpage
\subsection{Multi-Aquifer Well (MAW) Package}
\input{gwf/maw}

\newpage
\subsection{Streamflow Routing (SFR) Package}
Input to the Streamflow Routing (SFR) Package is read from the file that has type ``SFR6'' in the Name File. Any number of SFR Packages can be specified for a single groundwater flow model; however, water cannot be routed between reaches in separate packages except in cases where the MVR Package is used to route water between separate packages.

Reach connectivity must be explicitly specified for this version of the SFR Package, unlike the abbreviated SFR Package segment connectivity specified in previous versions of MODFLOW. Explicit specification of reach connectivity has been adopted to facilitate better validation of stream network connectivity by the program. Explicit reach connectivity means that a reach must be specified as an upstream connection for all downstream connections to the reach. Downstream connections for a reach are denoted with a negative reach number. Flow in a reach is unidirectional, always flowing from the upstream end to the downstream end of a reach. An example of the reach connectivity for a hypothetical stream network is shown in figure~\ref{fig:sfr-connectivity}.

\begin{figure}[ht]
	\centering
	\includegraphics[scale=1.0]{../Figures/sfr-connectivity}
	\caption[Illustration of a simple stream network having seven reaches with a junction having two reaches, a confluence of two reaches, and the resulting reach connectivity]{Simple stream network having seven reaches with a junction having two reaches, a confluence of two reaches, and the resulting reach connectivity. Downstream connections for a reach must include the reach as an upstream connection for all downstream connections to the reach. Downstream connections for a  reach are denoted with a negative reach number}
	\label{fig:sfr-connectivity}
\end{figure}


\vspace{5mm}
\subsubsection{Structure of Blocks}

\vspace{5mm}
\noindent \textit{FOR EACH SIMULATION}
\lstinputlisting[style=blockdefinition]{./mf6ivar/tex/gwf-sfr-options.dat}
\lstinputlisting[style=blockdefinition]{./mf6ivar/tex/gwf-sfr-dimensions.dat}
\lstinputlisting[style=blockdefinition]{./mf6ivar/tex/gwf-sfr-packagedata.dat}

\vspace{5mm}
\noindent \textit{CROSSSECTIONS BLOCK IS OPTIONAL}
\lstinputlisting[style=blockdefinition]{./mf6ivar/tex/gwf-sfr-crosssections.dat}

\lstinputlisting[style=blockdefinition]{./mf6ivar/tex/gwf-sfr-connectiondata.dat}

\vspace{5mm}
\noindent \textit{IF ndv IS GREATER THAN ZERO FOR ANY REACH}
\lstinputlisting[style=blockdefinition]{./mf6ivar/tex/gwf-sfr-diversions.dat}

\vspace{5mm}
\noindent \textit{FOR ANY STRESS PERIOD}
\lstinputlisting[style=blockdefinition]{./mf6ivar/tex/gwf-sfr-period.dat}
\advancedpackageperioddescription{reach}{reaches}

\vspace{5mm}
\subsubsection{Explanation of Variables}
\begin{description}
% DO NOT MODIFY THIS FILE DIRECTLY.  IT IS CREATED BY mf6ivar.py 

\item \textbf{Block: OPTIONS}

\begin{description}
\item \texttt{auxiliary}---defines an array of one or more auxiliary variable names.  There is no limit on the number of auxiliary variables that can be provided on this line; however, lists of information provided in subsequent blocks must have a column of data for each auxiliary variable name defined here.   The number of auxiliary variables detected on this line determines the value for naux.  Comments cannot be provided anywhere on this line as they will be interpreted as auxiliary variable names.  Auxiliary variables may not be used by the package, but they will be available for use by other parts of the program.  The program will terminate with an error if auxiliary variables are specified on more than one line in the options block.

\item \texttt{BOUNDNAMES}---keyword to indicate that boundary names may be provided with the list of stream reach cells.

\item \texttt{PRINT\_INPUT}---keyword to indicate that the list of stream reach information will be written to the listing file immediately after it is read.

\item \texttt{PRINT\_STAGE}---keyword to indicate that the list of stream reach stages will be printed to the listing file for every stress period in which ``HEAD PRINT'' is specified in Output Control.  If there is no Output Control option and PRINT\_STAGE is specified, then stages are printed for the last time step of each stress period.

\item \texttt{PRINT\_FLOWS}---keyword to indicate that the list of stream reach flow rates will be printed to the listing file for every stress period time step in which ``BUDGET PRINT'' is specified in Output Control.  If there is no Output Control option and ``PRINT\_FLOWS'' is specified, then flow rates are printed for the last time step of each stress period.

\item \texttt{SAVE\_FLOWS}---keyword to indicate that stream reach flow terms will be written to the file specified with ``BUDGET FILEOUT'' in Output Control.

\item \texttt{STAGE}---keyword to specify that record corresponds to stage.

\item \texttt{stagefile}---name of the binary output file to write stage information.

\item \texttt{BUDGET}---keyword to specify that record corresponds to the budget.

\item \texttt{FILEOUT}---keyword to specify that an output filename is expected next.

\item \texttt{budgetfile}---name of the binary output file to write budget information.

\item \texttt{BUDGETCSV}---keyword to specify that record corresponds to the budget CSV.

\item \texttt{budgetcsvfile}---name of the comma-separated value (CSV) output file to write budget summary information.  A budget summary record will be written to this file for each time step of the simulation.

\item \texttt{PACKAGE\_CONVERGENCE}---keyword to specify that record corresponds to the package convergence comma spaced values file.

\item \texttt{package\_convergence\_filename}---name of the comma spaced values output file to write package convergence information.

\item \texttt{TS6}---keyword to specify that record corresponds to a time-series file.

\item \texttt{FILEIN}---keyword to specify that an input filename is expected next.

\item \texttt{ts6\_filename}---defines a time-series file defining time series that can be used to assign time-varying values. See the ``Time-Variable Input'' section for instructions on using the time-series capability.

\item \texttt{OBS6}---keyword to specify that record corresponds to an observations file.

\item \texttt{obs6\_filename}---name of input file to define observations for the SFR package. See the ``Observation utility'' section for instructions for preparing observation input files. Tables \ref{table:gwf-obstypetable} and \ref{table:gwt-obstypetable} lists observation type(s) supported by the SFR package.

\item \texttt{MOVER}---keyword to indicate that this instance of the SFR Package can be used with the Water Mover (MVR) Package.  When the MOVER option is specified, additional memory is allocated within the package to store the available, provided, and received water.

\item \texttt{maximum\_picard\_iterations}---value that defines the maximum number of Streamflow Routing picard iterations allowed when solving for reach stages and flows as part of the GWF formulate step. Picard iterations are used to minimize differences in SFR package results between subsequent GWF picard (non-linear) iterations as a result of non-optimal reach numbering. If reaches are numbered in order, from upstream to downstream, MAXIMUM\_PICARD\_ITERATIONS can be set to 1 to reduce model run time. By default, MAXIMUM\_PICARD\_ITERATIONS is equal to 100.

\item \texttt{maximum\_iterations}---value that defines the maximum number of Streamflow Routing Newton-Raphson iterations allowed for a reach. By default, MAXIMUM\_ITERATIONS is equal to 100.

\item \texttt{maximum\_depth\_change}---value that defines the depth closure tolerance. By default, DMAXCHG is equal to $1 \times 10^{-5}$.

\item \texttt{unit\_conversion}---value (or conversion factor) that is used in calculating stream depth for stream reach. A constant of 1.486 is used for flow units of cubic feet per second, and a constant of 1.0 is used for units of cubic meters per second. The constant must be multiplied by 86,400 when using time units of days in the simulation.

\end{description}
\item \textbf{Block: DIMENSIONS}

\begin{description}
\item \texttt{nreaches}---integer value specifying the number of stream reaches.  There must be NREACHES entries in the PACKAGEDATA block.

\end{description}
\item \textbf{Block: PACKAGEDATA}

\begin{description}
\item \texttt{rno}---integer value that defines the reach number associated with the specified PACKAGEDATA data on the line. RNO must be greater than zero and less than or equal to NREACHES. Reach information must be specified for every reach or the program will terminate with an error.  The program will also terminate with an error if information for a reach is specified more than once.

\item \texttt{cellid}---The keyword `NONE' must be specified for reaches that are not connected to an underlying GWF cell. The keyword `NONE' is used for reaches that are in cells that have IDOMAIN values less than one or are in areas not covered by the GWF model grid. Reach-aquifer flow is not calculated if the keyword `NONE' is specified.

\item \texttt{rlen}---real value that defines the reach length. RLEN must be greater than zero.

\item \texttt{rwid}---real value that defines the reach width. RWID must be greater than zero.

\item \texttt{rgrd}---real value that defines the stream gradient (slope) across the reach. RGRD must be greater than zero.

\item \texttt{rtp}---real value that defines the top elevation of the reach streambed.

\item \texttt{rbth}---real value that defines the thickness of the reach streambed. RBTH can be any value if CELLID is `NONE'. Otherwise, RBTH must be greater than zero.

\item \texttt{rhk}---real value that defines the hydraulic conductivity of the reach streambed. RHK can be any positive value if CELLID is `NONE'. Otherwise, RHK must be greater than zero.

\item \textcolor{blue}{\texttt{man}---real or character value that defines the Manning's roughness coefficient for the reach. MAN must be greater than zero.  If the Options block includes a TIMESERIESFILE entry (see the ``Time-Variable Input'' section), values can be obtained from a time series by entering the time-series name in place of a numeric value.}

\item \texttt{ncon}---integer value that defines the number of reaches connected to the reach.  If a value of zero is specified for NCON an entry for RNO is still required in the subsequent CONNECTIONDATA block.

\item \textcolor{blue}{\texttt{ustrf}---real value that defines the fraction of upstream flow from each upstream reach that is applied as upstream inflow to the reach. The sum of all USTRF values for all reaches connected to the same upstream reach must be equal to one and USTRF must be greater than or equal to zero. If the Options block includes a TIMESERIESFILE entry (see the ``Time-Variable Input'' section), values can be obtained from a time series by entering the time-series name in place of a numeric value.}

\item \texttt{ndv}---integer value that defines the number of downstream diversions for the reach.

\item \textcolor{blue}{\texttt{aux}---represents the values of the auxiliary variables for each stream reach. The values of auxiliary variables must be present for each stream reach. The values must be specified in the order of the auxiliary variables specified in the OPTIONS block.  If the package supports time series and the Options block includes a TIMESERIESFILE entry (see the ``Time-Variable Input'' section), values can be obtained from a time series by entering the time-series name in place of a numeric value.}

\item \texttt{boundname}---name of the stream reach cell.  BOUNDNAME is an ASCII character variable that can contain as many as 40 characters.  If BOUNDNAME contains spaces in it, then the entire name must be enclosed within single quotes.

\end{description}
\item \textbf{Block: CROSSSECTIONS}

\begin{description}
\item \texttt{rno}---integer value that defines the reach number associated with the specified cross-section table file on the line. RNO must be greater than zero and less than or equal to NREACHES. The program will also terminate with an error if table information for a reach is specified more than once.

\item \texttt{TAB6}---keyword to specify that record corresponds to a cross-section table file.

\item \texttt{FILEIN}---keyword to specify that an input filename is expected next.

\item \texttt{tab6\_filename}---character string that defines the path and filename for the file containing cross-section table data for the reach. The TAB6\_FILENAME file includes the number of entries in the file and the station elevation data in terms of the fractional width and the reach depth. Instructions for creating the TAB6\_FILENAME input file are provided in SFR Reach Cross-Section Table Input File section.

\end{description}
\item \textbf{Block: CONNECTIONDATA}

\begin{description}
\item \texttt{rno}---integer value that defines the reach number associated with the specified CONNECTIONDATA data on the line. RNO must be greater than zero and less than or equal to NREACHES. Reach connection information must be specified for every reach or the program will terminate with an error.  The program will also terminate with an error if connection information for a reach is specified more than once.

\item \texttt{ic}---integer value that defines the reach number of the reach connected to the current reach and whether it is connected to the upstream or downstream end of the reach. Negative IC numbers indicate connected reaches are connected to the downstream end of the current reach. Positive IC numbers indicate connected reaches are connected to the upstream end of the current reach. The absolute value of IC must be greater than zero and less than or equal to NREACHES. IC should not be specified when NCON is zero but must be specified otherwise.

\end{description}
\item \textbf{Block: DIVERSIONS}

\begin{description}
\item \texttt{rno}---integer value that defines the reach number associated with the specified DIVERSIONS data on the line. RNO must be greater than zero and less than or equal to NREACHES.  Reach diversion information must be specified for every reach with a NDV value greater than 0 or the program will terminate with an error.  The program will also terminate with an error if diversion information for a given reach diversion is specified more than once.

\item \texttt{idv}---integer value that defines the downstream diversion number for the diversion for reach RNO. IDV must be greater than zero and less than or equal to NDV for reach RNO.

\item \texttt{iconr}---integer value that defines the downstream reach that will receive the diverted water. IDV must be greater than zero and less than or equal to NREACHES. Furthermore, reach  ICONR must be a downstream connection for reach RNO.

\item \texttt{cprior}---character string value that defines the the prioritization system for the diversion, such as when insufficient water is available to meet all diversion stipulations, and is used in conjunction with the value of FLOW value specified in the STRESS\_PERIOD\_DATA section. Available diversion options include:  (1) CPRIOR = `FRACTION', then the amount of the diversion is computed as a fraction of the streamflow leaving reach RNO ($Q_{DS}$); in this case, 0.0 $\le$ DIVFLOW $\le$ 1.0.  (2) CPRIOR = `EXCESS', a diversion is made only if $Q_{DS}$ for reach RNO exceeds the value of DIVFLOW. If this occurs, then the quantity of water diverted is the excess flow ($Q_{DS} -$ DIVFLOW) and $Q_{DS}$ from reach RNO is set equal to DIVFLOW. This represents a flood-control type of diversion, as described by Danskin and Hanson (2002). (3) CPRIOR = `THRESHOLD', then if $Q_{DS}$ in reach RNO is less than the specified diversion flow DIVFLOW, no water is diverted from reach RNO. If $Q_{DS}$ in reach RNO is greater than or equal to DIVFLOW, DIVFLOW is diverted and $Q_{DS}$ is set to the remainder ($Q_{DS} -$ DIVFLOW)). This approach assumes that once flow in the stream is sufficiently low, diversions from the stream cease, and is the `priority' algorithm that originally was programmed into the STR1 Package (Prudic, 1989).  (4) CPRIOR = `UPTO' -- if $Q_{DS}$ in reach RNO is greater than or equal to the specified diversion flow DIVFLOW, $Q_{DS}$ is reduced by DIVFLOW. If $Q_{DS}$ in reach RNO is less than DIVFLOW, DIVFLOW is set to $Q_{DS}$ and there will be no flow available for reaches connected to downstream end of reach RNO.

\end{description}
\item \textbf{Block: PERIOD}

\begin{description}
\item \texttt{iper}---integer value specifying the starting stress period number for which the data specified in the PERIOD block apply.  IPER must be less than or equal to NPER in the TDIS Package and greater than zero.  The IPER value assigned to a stress period block must be greater than the IPER value assigned for the previous PERIOD block.  The information specified in the PERIOD block will continue to apply for all subsequent stress periods, unless the program encounters another PERIOD block.

\item \texttt{rno}---integer value that defines the reach number associated with the specified PERIOD data on the line. RNO must be greater than zero and less than or equal to NREACHES.

\item \texttt{sfrsetting}---line of information that is parsed into a keyword and values.  Keyword values that can be used to start the SFRSETTING string include: STATUS, MANNING, STAGE, INFLOW, RAINFALL, EVAPORATION, RUNOFF, DIVERSION, UPSTREAM\_FRACTION, and AUXILIARY.

\begin{lstlisting}[style=blockdefinition]
STATUS <status>
MANNING <@manning@>
STAGE <@stage@>
INFLOW <@inflow@>
RAINFALL <@rainfall@>
EVAPORATION <@evaporation@>
RUNOFF <@runoff@>
DIVERSION <idv> <@divflow@> 
UPSTREAM_FRACTION <upstream_fraction>
CROSS_SECTION TAB6 FILEIN <tab6_filename> 
AUXILIARY <auxname> <@auxval@> 
\end{lstlisting}

\item \texttt{status}---keyword option to define stream reach status.  STATUS can be ACTIVE, INACTIVE, or SIMPLE. The SIMPLE STATUS option simulates streamflow using a user-specified stage for a reach or a stage set to the top of the reach (depth = 0). In cases where the simulated leakage calculated using the specified stage exceeds the sum of inflows to the reach, the stage is set to the top of the reach and leakage is set equal to the sum of inflows. Upstream fractions should be changed using the UPSTREAM\_FRACTION SFRSETTING if the status for one or more reaches is changed to ACTIVE or INACTIVE. For example, if one of two downstream connections for a reach is inactivated, the upstream fraction for the active and inactive downstream reach should be changed to 1.0 and 0.0, respectively, to ensure that the active reach receives all of the downstream outflow from the upstream reach. By default, STATUS is ACTIVE.

\item \textcolor{blue}{\texttt{manning}---real or character value that defines the Manning's roughness coefficient for the reach. MANNING must be greater than zero.  If the Options block includes a TIMESERIESFILE entry (see the ``Time-Variable Input'' section), values can be obtained from a time series by entering the time-series name in place of a numeric value.}

\item \textcolor{blue}{\texttt{stage}---real or character value that defines the stage for the reach. The specified STAGE is only applied if the reach uses the simple routing option. If STAGE is not specified for reaches that use the simple routing option, the specified stage is set to the top of the reach. If the Options block includes a TIMESERIESFILE entry (see the ``Time-Variable Input'' section), values can be obtained from a time series by entering the time-series name in place of a numeric value.}

\item \textcolor{blue}{\texttt{inflow}---real or character value that defines the volumetric inflow rate for the streamflow routing reach. If the Options block includes a TIMESERIESFILE entry (see the ``Time-Variable Input'' section), values can be obtained from a time series by entering the time-series name in place of a numeric value. By default, inflow rates are zero for each reach.}

\item \textcolor{blue}{\texttt{rainfall}---real or character value that defines the  volumetric rate per unit area of water added by precipitation directly on the streamflow routing reach. If the Options block includes a TIMESERIESFILE entry (see the ``Time-Variable Input'' section), values can be obtained from a time series by entering the time-series name in place of a numeric value. By default, rainfall  rates are zero for each reach.}

\item \textcolor{blue}{\texttt{evaporation}---real or character value that defines the volumetric rate per unit area of water subtracted by evaporation from the streamflow routing reach. A positive evaporation rate should be provided. If the Options block includes a TIMESERIESFILE entry (see the ``Time-Variable Input'' section), values can be obtained from a time series by entering the time-series name in place of a numeric value. If the volumetric evaporation rate for a reach exceeds the sources of water to the reach (upstream and specified inflows, rainfall, and runoff but excluding groundwater leakage into the reach) the volumetric evaporation rate is limited to the sources of water to the reach. By default, evaporation rates are zero for each reach.}

\item \textcolor{blue}{\texttt{runoff}---real or character value that defines the volumetric rate of diffuse overland runoff that enters the streamflow routing reach. If the Options block includes a TIMESERIESFILE entry (see the ``Time-Variable Input'' section), values can be obtained from a time series by entering the time-series name in place of a numeric value. If the volumetric runoff rate for a reach is negative and exceeds inflows to the reach (upstream and specified inflows, and rainfall but excluding groundwater leakage into the reach) the volumetric runoff rate is limited to inflows to the reach and the volumetric evaporation rate for the reach is set to zero. By default, runoff rates are zero for each reach.}

\item \texttt{DIVERSION}---keyword to indicate diversion record.

\item \texttt{idv}---an integer value specifying which diversion of reach RNO that DIVFLOW is being specified for.  Must be less or equal to ndv for the current reach (RNO).

\item \textcolor{blue}{\texttt{divflow}---real or character value that defines the volumetric diversion (DIVFLOW) rate for the streamflow routing reach. If the Options block includes a TIMESERIESFILE entry (see the ``Time-Variable Input'' section), values can be obtained from a time series by entering the time-series name in place of a numeric value.}

\item \texttt{upstream\_fraction}---real value that defines the fraction of upstream flow (USTRF) from each upstream reach that is applied as upstream inflow to the reach. The sum of all USTRF values for all reaches connected to the same upstream reach must be equal to one.

\item \texttt{CROSS\_SECTION}---keyword to specify that record corresponds to a reach cross-section.

\item \texttt{TAB6}---keyword to specify that record corresponds to a cross-section table file.

\item \texttt{FILEIN}---keyword to specify that an input filename is expected next.

\item \texttt{tab6\_filename}---character string that defines the path and filename for the file containing cross-section table data for the reach. The TAB6\_FILENAME file includes the number of entries in the file and the station elevation data in terms of the fractional width and the reach depth. Instructions for creating the TAB6\_FILENAME input file are provided in SFR Reach Cross-Section Table Input File section.

\item \texttt{AUXILIARY}---keyword for specifying auxiliary variable.

\item \texttt{auxname}---name for the auxiliary variable to be assigned AUXVAL.  AUXNAME must match one of the auxiliary variable names defined in the OPTIONS block. If AUXNAME does not match one of the auxiliary variable names defined in the OPTIONS block the data are ignored.

\item \textcolor{blue}{\texttt{auxval}---value for the auxiliary variable.  If the Options block includes a TIMESERIESFILE entry (see the ``Time-Variable Input'' section), values can be obtained from a time series by entering the time-series name in place of a numeric value.}

\end{description}


\end{description}

\vspace{5mm}
\subsubsection{Example Input File}
\lstinputlisting[style=inputfile]{./mf6ivar/examples/gwf-sfr-example.dat}

\vspace{5mm}
\subsubsection{Available observation types}
Streamflow Routing Package observations include reach stage and all of the terms that contribute to the continuity equation for each stream reach. Additional SFR Package observations include the sum of inflows from upstream reaches and from mover terms (\texttt{upstream-flow}) and downstream outflow from a reach prior to diversions and the mover package (\texttt{downstream-flow}). The data required for each SFR Package observation type is defined in table~\ref{table:gwf-sfrobstype}. Negative and positive values for \texttt{sfr} observations represent a loss from and gain to the GWF model, respectively. For all other flow terms, negative and positive values represent a loss from and gain from the SFR package, respectively.

\FloatBarrier
\begin{longtable}{p{2cm} p{2.75cm} p{2cm} p{1.25cm} p{7cm}}
\caption{Available SFR Package observation types} \tabularnewline

\hline
\hline
\textbf{Stress Package} & \textbf{Observation type} & \textbf{ID} & \textbf{ID2} & \textbf{Description} \\
\hline
\endfirsthead

\captionsetup{textformat=simple}
\caption*{\textbf{Table \arabic{table}.}{\quad}Available SFR Package observation types.---Continued} \\

\hline
\hline
\textbf{Stress Package} & \textbf{Observation type} & \textbf{ID} & \textbf{ID2} & \textbf{Description} \\
\hline
\endhead


\hline
\endfoot

SFR & stage & rno or boundname & -- & Surface-water stage in a stream-reach boundary. If boundname is specified, boundname must be unique for each reach. \\
SFR & ext-inflow & rno or boundname & -- & Inflow into a stream-reach from an external boundary for a stream-reach or a group of stream-reaches. \\
SFR & inflow & rno or boundname & -- & Inflow into a stream-reach from upstream reaches for a stream-reach or a group of stream-reaches. \\
SFR & from-mvr & rno or boundname & -- & Inflow into a stream-reach from the MVR package for a stream-reach or a group of stream-reaches. \\
SFR & rainfall & rno or boundname & -- & Rainfall rate applied to a stream-reach or a group of stream-reaches. \\
SFR & runoff & rno or boundname & -- & Runoff rate applied to a stream-reach or a group of stream-reaches. \\
SFR & sfr & rno or boundname & -- & Simulated flow rate for a stream-reach and its aquifer connection for a stream-reach or a group of stream-reaches. \\
SFR & evaporation & rno or boundname & -- & Simulated evaporation rate from a stream-reach or a group of stream-reaches. \\
SFR & outflow & rno or boundname & -- & Outflow from a stream-reach to downstream reaches for a stream-reach or a group of stream-reaches. \\
SFR & ext-outflow & rno or boundname & -- & Outflow from a stream-reach to an external boundary for a stream-reach or a group of stream-reaches. \\
SFR & to-mvr & rno or boundname & -- & Outflow from a stream-reach that is available for the MVR package for a stream-reach or a group of stream-reaches. \\
SFR & upstream-flow & rno or boundname & -- & Upstream flow for a stream-reach or a group of stream-reaches from upstream reaches and the MVR package. \\
SFR & downstream-flow & rno or boundname & -- & Downstream flow for a stream-reach or a group of stream-reaches prior to diversions and the MVR package. \\
SFR & depth & rno or boundname & -- & Surface-water depth in a stream-reach boundary. If boundname is specified, boundname must be unique for each reach. \\
SFR & wet-perimeter & rno or boundname & -- & Wetted perimeter in a stream-reach boundary. If boundname is specified, boundname must be unique for each reach. \\
SFR & wet-area & rno or boundname & -- & Wetted cross-section area in a stream-reach boundary. If boundname is specified, boundname must be unique for each reach. \\
SFR & wet-width & rno or boundname & -- & Wetted top width in a stream-reach boundary. If boundname is specified, boundname must be unique for each reach. \\


\label{table:gwf-sfrobstype}
\end{longtable}
\FloatBarrier

\vspace{5mm}
\subsubsection{Example Observation Input File}
\lstinputlisting[style=inputfile]{./mf6ivar/examples/gwf-sfr-example-obs.dat}

\newpage
\subsection{Streamflow Routing Package Cross-Sections Table Input File}
Cross-sections tables of distance and relative depth can be specified for individual reaches.  Cross-Section tables are specified by including file names in the LAKE\_TABLES block of the SFR Package.  These file names correspond to a lake table input file.  The format of the lake table input file is described here.

\vspace{5mm}
\subsubsection{Structure of Blocks}
\vspace{5mm}

\lstinputlisting[style=blockdefinition]{./mf6ivar/tex/utl-sfrtab-dimensions.dat}
\lstinputlisting[style=blockdefinition]{./mf6ivar/tex/utl-sfrtab-table.dat}
\vspace{5mm}

\vspace{5mm}
\subsubsection{Explanation of Variables}
\begin{description}
% DO NOT MODIFY THIS FILE DIRECTLY.  IT IS CREATED BY mf6ivar.py 

\item \textbf{Block: DIMENSIONS}

\begin{description}
\item \texttt{nrow}---integer value specifying the number of rows in the reach cross-section table. There must be NROW rows of data in the TABLE block.

\item \texttt{ncol}---integer value specifying the number of columns in the reach cross-section table. There must be NCOL columns of data in the TABLE block. Currently, NCOL must be equal to 2.

\end{description}
\item \textbf{Block: TABLE}

\begin{description}
\item \texttt{xfraction}---real value that defines the station (x) data for the cross-section as a fraction of the width (RWID) of the reach.

\item \texttt{depth}---real value that defines the elevation (z) data for the cross-section as a depth relative to the top elevation of the reach (RTP) and corresponding to the station data on the same line.

\end{description}


\end{description}

\subsubsection{Example Input File}
%\lstinputlisting[style=inputfile]{./mf6ivar/examples/utl-sfrtab-example.dat}



\newpage
\subsection{Lake (LAK) Package}
Input to the Lake (LAK) Package is read from the file that has type ``LAK6'' in the Name File.  Any number of LAK Packages can be specified for a single groundwater flow model.

\vspace{5mm}
\subsubsection{Structure of Blocks}
\vspace{5mm}

\noindent \textit{FOR EACH SIMULATION}
\lstinputlisting[style=blockdefinition]{./mf6ivar/tex/gwf-lak-options.dat}
\lstinputlisting[style=blockdefinition]{./mf6ivar/tex/gwf-lak-dimensions.dat}
\lstinputlisting[style=blockdefinition]{./mf6ivar/tex/gwf-lak-packagedata.dat}
\noindent \textit{IF \texttt{nlakeconn} IS GREATER THAN ZERO FOR ANY LAKE}
\lstinputlisting[style=blockdefinition]{./mf6ivar/tex/gwf-lak-connectiondata.dat}
\noindent \textit{IF \texttt{ntables} IS GREATER THAN ZERO}
\lstinputlisting[style=blockdefinition]{./mf6ivar/tex/gwf-lak-tables.dat}
\noindent \textit{IF \texttt{noutlets} IS GREATER THAN ZERO FOR ANY LAKE}
\lstinputlisting[style=blockdefinition]{./mf6ivar/tex/gwf-lak-outlets.dat}

\vspace{5mm}
\noindent \textit{FOR ANY STRESS PERIOD}
\lstinputlisting[style=blockdefinition]{./mf6ivar/tex/gwf-lak-period.dat}
\advancedpackageperioddescription{lake}{lakes}

\vspace{5mm}
\subsubsection{Explanation of Variables}
\begin{description}
% DO NOT MODIFY THIS FILE DIRECTLY.  IT IS CREATED BY mf6ivar.py 

\item \textbf{Block: OPTIONS}

\begin{description}
\item \texttt{auxiliary}---defines an array of one or more auxiliary variable names.  There is no limit on the number of auxiliary variables that can be provided on this line; however, lists of information provided in subsequent blocks must have a column of data for each auxiliary variable name defined here.   The number of auxiliary variables detected on this line determines the value for naux.  Comments cannot be provided anywhere on this line as they will be interpreted as auxiliary variable names.  Auxiliary variables may not be used by the package, but they will be available for use by other parts of the program.  The program will terminate with an error if auxiliary variables are specified on more than one line in the options block.

\item \texttt{BOUNDNAMES}---keyword to indicate that boundary names may be provided with the list of lake cells.

\item \texttt{PRINT\_INPUT}---keyword to indicate that the list of lake information will be written to the listing file immediately after it is read.

\item \texttt{PRINT\_STAGE}---keyword to indicate that the list of lake stages will be printed to the listing file for every stress period in which ``HEAD PRINT'' is specified in Output Control.  If there is no Output Control option and PRINT\_STAGE is specified, then stages are printed for the last time step of each stress period.

\item \texttt{PRINT\_FLOWS}---keyword to indicate that the list of lake flow rates will be printed to the listing file for every stress period time step in which ``BUDGET PRINT'' is specified in Output Control.  If there is no Output Control option and ``PRINT\_FLOWS'' is specified, then flow rates are printed for the last time step of each stress period.

\item \texttt{SAVE\_FLOWS}---keyword to indicate that lake flow terms will be written to the file specified with ``BUDGET FILEOUT'' in Output Control.

\item \texttt{STAGE}---keyword to specify that record corresponds to stage.

\item \texttt{stagefile}---name of the binary output file to write stage information.

\item \texttt{BUDGET}---keyword to specify that record corresponds to the budget.

\item \texttt{FILEOUT}---keyword to specify that an output filename is expected next.

\item \texttt{budgetfile}---name of the binary output file to write budget information.

\item \texttt{BUDGETCSV}---keyword to specify that record corresponds to the budget CSV.

\item \texttt{budgetcsvfile}---name of the comma-separated value (CSV) output file to write budget summary information.  A budget summary record will be written to this file for each time step of the simulation.

\item \texttt{PACKAGE\_CONVERGENCE}---keyword to specify that record corresponds to the package convergence comma spaced values file.

\item \texttt{package\_convergence\_filename}---name of the comma spaced values output file to write package convergence information.

\item \texttt{TS6}---keyword to specify that record corresponds to a time-series file.

\item \texttt{FILEIN}---keyword to specify that an input filename is expected next.

\item \texttt{ts6\_filename}---defines a time-series file defining time series that can be used to assign time-varying values. See the ``Time-Variable Input'' section for instructions on using the time-series capability.

\item \texttt{OBS6}---keyword to specify that record corresponds to an observations file.

\item \texttt{obs6\_filename}---name of input file to define observations for the LAK package. See the ``Observation utility'' section for instructions for preparing observation input files. Tables \ref{table:gwf-obstypetable} and \ref{table:gwt-obstypetable} lists observation type(s) supported by the LAK package.

\item \texttt{MOVER}---keyword to indicate that this instance of the LAK Package can be used with the Water Mover (MVR) Package.  When the MOVER option is specified, additional memory is allocated within the package to store the available, provided, and received water.

\item \texttt{surfdep}---real value that defines the surface depression depth for VERTICAL lake-GWF connections. If specified, SURFDEP must be greater than or equal to zero. If SURFDEP is not specified, a default value of zero is used for all vertical lake-GWF connections.

\item \texttt{time\_conversion}---value that is used in converting outlet flow terms that use Manning's equation or gravitational acceleration to consistent time units. TIME\_CONVERSION should be set to 1.0, 60.0, 3,600.0, 86,400.0, and 31,557,600.0 when using time units (TIME\_UNITS) of seconds, minutes, hours, days, or years in the simulation, respectively. CONVTIME does not need to be specified if no lake outlets are specified or TIME\_UNITS are seconds.

\item \texttt{length\_conversion}---real value that is used in converting outlet flow terms that use Manning's equation or gravitational acceleration to consistent length units. LENGTH\_CONVERSION should be set to 3.28081, 1.0, and 100.0 when using length units (LENGTH\_UNITS) of feet, meters, or centimeters in the simulation, respectively. LENGTH\_CONVERSION does not need to be specified if no lake outlets are specified or LENGTH\_UNITS are meters.

\end{description}
\item \textbf{Block: DIMENSIONS}

\begin{description}
\item \texttt{nlakes}---value specifying the number of lakes that will be simulated for all stress periods.

\item \texttt{noutlets}---value specifying the number of outlets that will be simulated for all stress periods. If NOUTLETS is not specified, a default value of zero is used.

\item \texttt{ntables}---value specifying the number of lakes tables that will be used to define the lake stage, volume relation, and surface area. If NTABLES is not specified, a default value of zero is used.

\end{description}
\item \textbf{Block: PACKAGEDATA}

\begin{description}
\item \texttt{lakeno}---integer value that defines the lake number associated with the specified PACKAGEDATA data on the line. LAKENO must be greater than zero and less than or equal to NLAKES. Lake information must be specified for every lake or the program will terminate with an error.  The program will also terminate with an error if information for a lake is specified more than once.

\item \texttt{strt}---real value that defines the starting stage for the lake.

\item \texttt{nlakeconn}---integer value that defines the number of GWF cells connected to this (LAKENO) lake. There can only be one vertical lake connection to each GWF cell. NLAKECONN must be greater than zero.

\item \textcolor{blue}{\texttt{aux}---represents the values of the auxiliary variables for each lake. The values of auxiliary variables must be present for each lake. The values must be specified in the order of the auxiliary variables specified in the OPTIONS block.  If the package supports time series and the Options block includes a TIMESERIESFILE entry (see the ``Time-Variable Input'' section), values can be obtained from a time series by entering the time-series name in place of a numeric value.}

\item \texttt{boundname}---name of the lake cell.  BOUNDNAME is an ASCII character variable that can contain as many as 40 characters.  If BOUNDNAME contains spaces in it, then the entire name must be enclosed within single quotes.

\end{description}
\item \textbf{Block: CONNECTIONDATA}

\begin{description}
\item \texttt{lakeno}---integer value that defines the lake number associated with the specified CONNECTIONDATA data on the line. LAKENO must be greater than zero and less than or equal to NLAKES. Lake connection information must be specified for every lake connection to the GWF model (NLAKECONN) or the program will terminate with an error.  The program will also terminate with an error if connection information for a lake connection to the GWF model is specified more than once.

\item \texttt{iconn}---integer value that defines the GWF connection number for this lake connection entry. ICONN must be greater than zero and less than or equal to NLAKECONN for lake LAKENO.

\item \texttt{cellid}---is the cell identifier, and depends on the type of grid that is used for the simulation.  For a structured grid that uses the DIS input file, CELLID is the layer, row, and column.   For a grid that uses the DISV input file, CELLID is the layer and CELL2D number.  If the model uses the unstructured discretization (DISU) input file, CELLID is the node number for the cell.

\item \texttt{claktype}---character string that defines the lake-GWF connection type for the lake connection. Possible lake-GWF connection type strings include:  VERTICAL--character keyword to indicate the lake-GWF connection is vertical  and connection conductance calculations use the hydraulic conductivity corresponding to the $K_{33}$ tensor component defined for CELLID in the NPF package. HORIZONTAL--character keyword to indicate the lake-GWF connection is horizontal and connection conductance calculations use the hydraulic conductivity corresponding to the $K_{11}$ tensor component defined for CELLID in the NPF package. EMBEDDEDH--character keyword to indicate the lake-GWF connection is embedded in a single cell and connection conductance calculations use the hydraulic conductivity corresponding to the $K_{11}$ tensor component defined for CELLID in the NPF package. EMBEDDEDV--character keyword to indicate the lake-GWF connection is embedded in a single cell and connection conductance calculations use the hydraulic conductivity corresponding to the $K_{33}$ tensor component defined for CELLID in the NPF package. Embedded lakes can only be connected to a single cell (NLAKECONN = 1) and there must be a lake table associated with each embedded lake.

\item \texttt{bedleak}---character string or real value that defines the bed leakance for the lake-GWF connection. BEDLEAK must be greater than or equal to zero or specified to be NONE. If BEDLEAK is specified to be NONE, the lake-GWF connection conductance is solely a function of aquifer properties in the connected GWF cell and lakebed sediments are assumed to be absent.

\item \texttt{belev}---real value that defines the bottom elevation for a HORIZONTAL lake-GWF connection. Any value can be specified if CLAKTYPE is VERTICAL, EMBEDDEDH, or EMBEDDEDV. If CLAKTYPE is HORIZONTAL and BELEV is not equal to TELEV, BELEV must be greater than or equal to the bottom of the GWF cell CELLID. If BELEV is equal to TELEV, BELEV is reset to the bottom of the GWF cell CELLID.

\item \texttt{telev}---real value that defines the top elevation for a HORIZONTAL lake-GWF connection. Any value can be specified if CLAKTYPE is VERTICAL, EMBEDDEDH, or EMBEDDEDV. If CLAKTYPE is HORIZONTAL and TELEV is not equal to BELEV, TELEV must be less than or equal to the top of the GWF cell CELLID. If TELEV is equal to BELEV, TELEV is reset to the top of the GWF cell CELLID.

\item \texttt{connlen}---real value that defines the distance between the connected GWF CELLID node and the lake for a HORIZONTAL, EMBEDDEDH, or EMBEDDEDV lake-GWF connection. CONLENN must be greater than zero for a HORIZONTAL, EMBEDDEDH, or EMBEDDEDV lake-GWF connection. Any value can be specified if CLAKTYPE is VERTICAL.

\item \texttt{connwidth}---real value that defines the connection face width for a HORIZONTAL lake-GWF connection. CONNWIDTH must be greater than zero for a HORIZONTAL lake-GWF connection. Any value can be specified if CLAKTYPE is VERTICAL, EMBEDDEDH, or EMBEDDEDV.

\end{description}
\item \textbf{Block: TABLES}

\begin{description}
\item \texttt{lakeno}---integer value that defines the lake number associated with the specified TABLES data on the line. LAKENO must be greater than zero and less than or equal to NLAKES. The program will terminate with an error if table information for a lake is specified more than once or the number of specified tables is less than NTABLES.

\item \texttt{TAB6}---keyword to specify that record corresponds to a table file.

\item \texttt{FILEIN}---keyword to specify that an input filename is expected next.

\item \texttt{tab6\_filename}---character string that defines the path and filename for the file containing lake table data for the lake connection. The TAB6\_FILENAME file includes the number of entries in the file and the relation between stage, volume, and surface area for each entry in the file. Lake table files for EMBEDDEDH and EMBEDDEDV lake-GWF connections also include lake-GWF exchange area data for each entry in the file. Instructions for creating the TAB6\_FILENAME input file are provided in Lake Table Input File section.

\end{description}
\item \textbf{Block: OUTLETS}

\begin{description}
\item \texttt{outletno}---integer value that defines the outlet number associated with the specified OUTLETS data on the line. OUTLETNO must be greater than zero and less than or equal to NOUTLETS. Outlet information must be specified for every outlet or the program will terminate with an error. The program will also terminate with an error if information for a outlet is specified more than once.

\item \texttt{lakein}---integer value that defines the lake number that outlet is connected to. LAKEIN must be greater than zero and less than or equal to NLAKES.

\item \texttt{lakeout}---integer value that defines the lake number that outlet discharge from lake outlet OUTLETNO is routed to. LAKEOUT must be greater than or equal to zero and less than or equal to NLAKES. If LAKEOUT is zero, outlet discharge from lake outlet OUTLETNO is discharged to an external boundary.

\item \texttt{couttype}---character string that defines the outlet type for the outlet OUTLETNO. Possible COUTTYPE strings include: SPECIFIED--character keyword to indicate the outlet is defined as a specified flow.  MANNING--character keyword to indicate the outlet is defined using Manning's equation. WEIR--character keyword to indicate the outlet is defined using a sharp weir equation.

\item \textcolor{blue}{\texttt{invert}---real value that defines the invert elevation for the lake outlet. Any value can be specified if COUTTYPE is SPECIFIED. If the Options block includes a TIMESERIESFILE entry (see the ``Time-Variable Input'' section), values can be obtained from a time series by entering the time-series name in place of a numeric value.}

\item \textcolor{blue}{\texttt{width}---real value that defines the width of the lake outlet. Any value can be specified if COUTTYPE is SPECIFIED. If the Options block includes a TIMESERIESFILE entry (see the ``Time-Variable Input'' section), values can be obtained from a time series by entering the time-series name in place of a numeric value.}

\item \textcolor{blue}{\texttt{rough}---real value that defines the roughness coefficient for the lake outlet. Any value can be specified if COUTTYPE is not MANNING. If the Options block includes a TIMESERIESFILE entry (see the ``Time-Variable Input'' section), values can be obtained from a time series by entering the time-series name in place of a numeric value.}

\item \textcolor{blue}{\texttt{slope}---real value that defines the bed slope for the lake outlet. Any value can be specified if COUTTYPE is not MANNING. If the Options block includes a TIMESERIESFILE entry (see the ``Time-Variable Input'' section), values can be obtained from a time series by entering the time-series name in place of a numeric value.}

\end{description}
\item \textbf{Block: PERIOD}

\begin{description}
\item \texttt{iper}---integer value specifying the starting stress period number for which the data specified in the PERIOD block apply.  IPER must be less than or equal to NPER in the TDIS Package and greater than zero.  The IPER value assigned to a stress period block must be greater than the IPER value assigned for the previous PERIOD block.  The information specified in the PERIOD block will continue to apply for all subsequent stress periods, unless the program encounters another PERIOD block.

\item \texttt{number}---integer value that defines the lake or outlet number associated with the specified PERIOD data on the line.  NUMBER must be greater than zero and less than or equal to NLAKES for a lake number and less than or equal to NOUTLETS for an outlet number.

\item \texttt{laksetting}---line of information that is parsed into a keyword and values.  Keyword values that can be used to start the LAKSETTING string include both keywords for lake settings and keywords for outlet settings.  Keywords for lake settings include: STATUS, STAGE, RAINFALL, EVAPORATION, RUNOFF, INFLOW, WITHDRAWAL, and AUXILIARY.  Keywords for outlet settings include RATE, INVERT, WIDTH, SLOPE, and ROUGH.

\begin{lstlisting}[style=blockdefinition]
STATUS <status>
STAGE <@stage@>
RAINFALL <@rainfall@>
EVAPORATION <@evaporation@>
RUNOFF <@runoff@>
INFLOW <@inflow@>
WITHDRAWAL <@withdrawal@>
RATE <@rate@>
INVERT <@invert@>
WIDTH <@width@>
SLOPE <@slope@>
ROUGH <@rough@>
AUXILIARY <auxname> <@auxval@> 
\end{lstlisting}

\item \texttt{status}---keyword option to define lake status.  STATUS can be ACTIVE, INACTIVE, or CONSTANT. By default, STATUS is ACTIVE.

\item \textcolor{blue}{\texttt{stage}---real or character value that defines the stage for the lake. The specified STAGE is only applied if the lake is a constant stage lake. If the Options block includes a TIMESERIESFILE entry (see the ``Time-Variable Input'' section), values can be obtained from a time series by entering the time-series name in place of a numeric value.}

\item \textcolor{blue}{\texttt{rainfall}---real or character value that defines the rainfall rate $(LT^{-1})$ for the lake. Value must be greater than or equal to zero. If the Options block includes a TIMESERIESFILE entry (see the ``Time-Variable Input'' section), values can be obtained from a time series by entering the time-series name in place of a numeric value.}

\item \textcolor{blue}{\texttt{evaporation}---real or character value that defines the maximum evaporation rate $(LT^{-1})$ for the lake. Value must be greater than or equal to zero. If the Options block includes a TIMESERIESFILE entry (see the ``Time-Variable Input'' section), values can be obtained from a time series by entering the time-series name in place of a numeric value.}

\item \textcolor{blue}{\texttt{runoff}---real or character value that defines the runoff rate $(L^3 T^{-1})$ for the lake. Value must be greater than or equal to zero. If the Options block includes a TIMESERIESFILE entry (see the ``Time-Variable Input'' section), values can be obtained from a time series by entering the time-series name in place of a numeric value.}

\item \textcolor{blue}{\texttt{inflow}---real or character value that defines the volumetric inflow rate $(L^3 T^{-1})$ for the lake. Value must be greater than or equal to zero. If the Options block includes a TIMESERIESFILE entry (see the ``Time-Variable Input'' section), values can be obtained from a time series by entering the time-series name in place of a numeric value. By default, inflow rates are zero for each lake.}

\item \textcolor{blue}{\texttt{withdrawal}---real or character value that defines the maximum withdrawal rate $(L^3 T^{-1})$ for the lake. Value must be greater than or equal to zero. If the Options block includes a TIMESERIESFILE entry (see the ``Time-Variable Input'' section), values can be obtained from a time series by entering the time-series name in place of a numeric value.}

\item \textcolor{blue}{\texttt{rate}---real or character value that defines the extraction rate for the lake outflow. A positive value indicates inflow and a negative value indicates outflow from the lake. RATE only applies to active (IBOUND $>$ 0) lakes. A specified RATE is only applied if COUTTYPE for the OUTLETNO is SPECIFIED. If the Options block includes a TIMESERIESFILE entry (see the ``Time-Variable Input'' section), values can be obtained from a time series by entering the time-series name in place of a numeric value. By default, the RATE for each SPECIFIED lake outlet is zero.}

\item \textcolor{blue}{\texttt{invert}---real or character value that defines the invert elevation for the lake outlet. A specified INVERT value is only used for active lakes if COUTTYPE for lake outlet OUTLETNO is not SPECIFIED. If the Options block includes a TIMESERIESFILE entry (see the ``Time-Variable Input'' section), values can be obtained from a time series by entering the time-series name in place of a numeric value.}

\item \textcolor{blue}{\texttt{rough}---real value that defines the roughness coefficient for the lake outlet. Any value can be specified if COUTTYPE is not MANNING. If the Options block includes a TIMESERIESFILE entry (see the ``Time-Variable Input'' section), values can be obtained from a time series by entering the time-series name in place of a numeric value.}

\item \textcolor{blue}{\texttt{width}---real or character value that defines the width of the lake outlet. A specified WIDTH value is only used for active lakes if COUTTYPE for lake outlet OUTLETNO is not SPECIFIED. If the Options block includes a TIMESERIESFILE entry (see the ``Time-Variable Input'' section), values can be obtained from a time series by entering the time-series name in place of a numeric value.}

\item \textcolor{blue}{\texttt{slope}---real or character value that defines the bed slope for the lake outlet. A specified SLOPE value is only used for active lakes if COUTTYPE for lake outlet OUTLETNO is MANNING. If the Options block includes a TIMESERIESFILE entry (see the ``Time-Variable Input'' section), values can be obtained from a time series by entering the time-series name in place of a numeric value.}

\item \texttt{AUXILIARY}---keyword for specifying auxiliary variable.

\item \texttt{auxname}---name for the auxiliary variable to be assigned AUXVAL.  AUXNAME must match one of the auxiliary variable names defined in the OPTIONS block. If AUXNAME does not match one of the auxiliary variable names defined in the OPTIONS block the data are ignored.

\item \textcolor{blue}{\texttt{auxval}---value for the auxiliary variable. If the Options block includes a TIMESERIESFILE entry (see the ``Time-Variable Input'' section), values can be obtained from a time series by entering the time-series name in place of a numeric value.}

\end{description}


\end{description}

\vspace{5mm}
\subsubsection{Example Input File}
\lstinputlisting[style=inputfile]{./mf6ivar/examples/gwf-lak-example.dat}

\vspace{5mm}
\subsubsection{Available observation types}
Lake Package observations include lake stage and all of the terms that contribute to the continuity equation for each lake. Additional LAK Package observations include flow rates for individual outlets, lakes, or groups of lakes (\texttt{outlet}); the lake volume (\texttt{volume}); lake surface area (\texttt{surface-area}); wetted area for a lake-aquifer connection (\texttt{wetted-area}); and the conductance for a lake-aquifer connection conductance (\texttt{conductance}). The data required for each LAK Package observation type is defined in table~\ref{table:gwf-lakobstype}. Negative and positive values for \texttt{lak} observations represent a loss from and gain to the GWF model, respectively. For all other flow terms, negative and positive values represent a loss from and gain from the LAK package, respectively.

\begin{longtable}{p{2cm} p{2.75cm} p{2cm} p{1.25cm} p{7cm}}
\caption{Available LAK Package observation types} \tabularnewline

\hline
\hline
\textbf{Stress Package} & \textbf{Observation type} & \textbf{ID} & \textbf{ID2} & \textbf{Description} \\
\hline
\endfirsthead

\captionsetup{textformat=simple}
\caption*{\textbf{Table \arabic{table}.}{\quad}Available LAK Package observation types.---Continued} \tabularnewline

\hline
\hline
\textbf{Stress Package} & \textbf{Observation type} & \textbf{ID} & \textbf{ID2} & \textbf{Description} \\
\hline
\endhead


\hline
\endfoot

\input{../Common/gwf-lakobs.tex}
\label{table:gwf-lakobstype}
\end{longtable}

\vspace{5mm}
\subsubsection{Example Observation Input File}
\lstinputlisting[style=inputfile]{./mf6ivar/examples/gwf-lak-example-obs.dat}

\newpage
\subsection{Lake Table Input File}
Lake tables of stage, volume, and surface area can be specified for individual lakes.  Lake tables are specified by including file names in the LAKE\_TABLES block of the LAK Package.  These file names correspond to a lake table input file.  The format of the lake table input file is described here.

\vspace{5mm}
\subsubsection{Structure of Blocks}
\vspace{5mm}

\lstinputlisting[style=blockdefinition]{./mf6ivar/tex/utl-laktab-dimensions.dat}
\lstinputlisting[style=blockdefinition]{./mf6ivar/tex/utl-laktab-table.dat}
\vspace{5mm}

\vspace{5mm}
\subsubsection{Explanation of Variables}
\begin{description}
% DO NOT MODIFY THIS FILE DIRECTLY.  IT IS CREATED BY mf6ivar.py 

\item \textbf{Block: DIMENSIONS}

\begin{description}
\item \texttt{nrow}---integer value specifying the number of rows in the lake table. There must be NROW rows of data in the TABLE block.

\item \texttt{ncol}---integer value specifying the number of columns in the lake table. There must be NCOL columns of data in the TABLE block. For lakes with HORIZONTAL and/or VERTICAL CTYPE connections, NCOL must be equal to 3. For lakes with EMBEDDEDH or EMBEDDEDV CTYPE connections, NCOL must be equal to 4.

\end{description}
\item \textbf{Block: TABLE}

\begin{description}
\item \texttt{stage}---real value that defines the stage corresponding to the remaining data on the line.

\item \texttt{volume}---real value that defines the lake volume corresponding to the stage specified on the line.

\item \texttt{sarea}---real value that defines the lake surface area corresponding to the stage specified on the line.

\item \texttt{barea}---real value that defines the lake-GWF exchange area corresponding to the stage specified on the line. BAREA is only specified if the CLAKTYPE for the lake is EMBEDDEDH or EMBEDDEDV.

\end{description}


\end{description}

\subsubsection{Example Input File}
\lstinputlisting[style=inputfile]{./mf6ivar/examples/utl-laktab-example.dat}



\newpage
\subsection{Unsaturated Zone Flow (UZF) Package}
\input{gwf/uzf}

\newpage
\subsection{Water Mover (MVR) Package}
\input{gwf/mvr}

\newpage
\subsection{Ghost-Node Correction (GNC) Package}
\input{gwf/gnc}

\newpage
\subsection{Groundwater Flow (GWF) Exchange}
\input{gwf/gwf-gwf}

