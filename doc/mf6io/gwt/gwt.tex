The GWT Model simulates three-dimensional transport of a single solute species in flowing groundwater.  The GWT Model solves the solute transport equation using numerical methods and a generalized control-volume finite-difference approach, which can be used with regular MODFLOW grids (DIS Package) or with unstructured grids (DISV and DISU Packages).  The GWT Model is designed to work with most of the new capabilities released with the GWF Model, including the Newton flow formulation, unstructured grids, advanced packages, and the movement of water between packages.  The GWF and GWT Models operate simultaneously during a \mf simulation to represent coupled groundwater flow and solute transport.  The GWT Model can also run separately from a GWF Model by reading the heads and flows saved by a previously run GWF Model.  The GWT model is also capable of working with the flows from another groundwater flow model, as long as the flows from that model can be written in the correct form to flow and head files.  

The purpose of the GWT Model is to calculate changes in solute concentration in both space and time.  Solute concentrations within an aquifer can change in response to multiple solute transport processes.  These processes include (1) advective transport of solute with flowing groundwater, (2) the combined hydrodynamic dispersion processes of velocity-dependent mechanical dispersion and chemical diffusion, (3) sorption of solutes by the aquifer matrix either by adsorption to individual solid grains or by absorbtion into solid grains, (4) transfer of solute into very low permeability aquifer material (called an immobile domain) where it can be stored and later released, (5) first- or zero-order solute decay or production in response to chemical or biological reactions, (6) mixing with fluids from groundwater sources and sinks, and (7) direct addition of solute mass.

With the present implementation, there can be multiple domains and multiple phases.  There is a single mobile domain, which normally consists of flowing groundwater, and there can be one or more immobile domains.  The GWT Model simulates the dissolved phase of chemical constituents in both the mobile and immobile domains.  The dissolved phase is also referred to in this report as the aqueous phase.  If sorption is represented, then the GWT Model also simulates the solid phase of the chemical constituent in both the mobile and immobile domains.  The dissolved and solid phases of the chemical constituent are tracked in the different domains by the GWT Model and can be reported as output as requested by the user.

This section describes the data files for a \mf Groundwater Transport (GWT) Model.  A GWT Model is added to the simulation by including a GWT entry in the MODELS block of the simulation name file.  There are three types of spatial discretization approaches that can be used with the GWT Model: DIS, DISV, and DISU.  The input instructions for these three packages are not described here in this section on GWT Model input; input instructions for these three packages are described in the section on GWF Model input.

The GWT Model is designed to permit input to be gathered, as it is needed, from many different files.  Likewise, results from the model calculations can be written to a number of output files. The GWT Model Listing File is a key file to which the GWT model output is written.  As \mf runs, information about the GWT Model is written to the GWT Model Listing File, including much of the input data (as a record of the simulation) and calculated results.  Details about the files used by each package are provided in this section on the GWT Model Instructions.

The GWT Model reads a file called the Name File, which specifies most of the files that will be used in a simulation. Several files are always required whereas other files are optional depending on the simulation. The Output Control Package receives instructions from the user to control the amount and frequency of output.  Details about the Name File and the Output Control Package are described in this section.

For the GWT Model, ``flows'' (unless stated otherwise) represent solute mass ``flow'' in mass per time, rather than groundwater flow.  

\subsection{Information for Existing Solute Transport Modelers}
The \mf GWT Model contains most of the functionality of MODFLOW-GWT, MT3DMS, MT3D-USGS and MODFLOW-USG.  The following list summarizes major differences between the GWT Model in \mf and previous MODFLOW-based solute transport programs.

\begin{enumerate}

\item The GWT Model simulates transport of a single chemical species; however, because \mf allows for multiple models of the same type to be included in a single simulation, multiple species can be represented by using multiple GWT Models.

\item There is no specialized flow and transport link file \citep{zheng2001modflow} used to pass the simulated groundwater flows to the transport model.  Instead, simulated flows from the GWF Model are passed in memory to the GWT Model while the program is running.  Alternatively, the GWT Model can read binary flow and head files saved by the GWF Model while it is running.  If the user intends to simulate transport through the advanced stress packages and Water Mover Package, then flows from these advanced packages must also be saved to binary files.  Names for these binary files are provided as input to the FMI Package.

\item The GWT Model is based on a generalized control-volume finite-difference method, which means that solute transport can be simulated using regular MODFLOW grids consisting of layers, rows, and columns, or solute transport can be simulated using unstructured grids.

\item Advection can be simulated using central-in-space weighting, upstream weighting, or an implicit second-order TVD scheme.  The GWT model does not have the Method of Characteristics (particle-based approaches) or an explicit TVD scheme.  Consequently, the GWT Model may require a higher level of spatial discretization than other transport models that use higher order terms for advection dominated systems.  This can be an important limitation for some problems, which require the preservation of sharp solute fronts. 

\item Variable-density flow and transport can be simulated by including a GWF Model and a GWT Model in the same \mf simulation.  The Buoyancy Package should be activated for the GWF Model so that fluid density is calculated as a function of simulated concentration.  If more than one chemical species is represented then the Buoyancy Package allows the simulated concentration for each of them to be used in the density equation of state.   \cite{langevin2020hydraulic} describe the hydraulic-head formation that is implemented in the Buoyancy Package for variable-density groundwater flow and present the results from \mf variable-density simulations.  The variable-density capabilities available in \mf replicate and extend the capabilities available in SEAWAT to include the Newton flow formulation and unstructured grids, for example.  

\item The GWT Model has a Source and Sink Mixing (SSM) Package for representing the effects of GWF stress package inflows and outflows on simulated concentrations.  There are two ways in which users can assign concentrations to the individual features in these stress package.  The first way is to activate a concentration auxiliary variable in the GWF stress package.  In the SSM input file, the user provides the name of the auxiliary variable to be used for concentration.  The second way is to create a special SSMI file, which contains user-assigned time-varying concentrations for stress package boundaries.

\item The GWT model includes the MST and IST Packages.  These two package collectively comprise the capabilities of the MT3DMS Reactions Package.

\item The MST Package contains the linear, Freundlich, and Langmuir isotherms for representing sorption.  The IST Packages contains only the linear isotherm for representation of sorption. 

\item The GWT model was designed so that the user can specify as many immobile domains and necessary to represent observed contaminant transport patterns and solute breakthrough curves.  The effects of an immobile domain are represented using the Immobile Storage and Transfer (IST) Package, and the user can specify as many IST Packages as necessary.  

\item Although there is GWF-GWF Exchange, a GWT-GWT Exchange has not yet been developed to connect multiple transport models, as might be done in a nested grid configuration.  

\item There is no option to automatically run the GWT Model to steady state using a single time step.  This is an option available in MT3DMS \citep{zheng2010supplemental}.  Steady state conditions must be determined by running the transport model under transient conditions until concentrations stabilize.

\item The GWT Model described in this report is capable of simulating solute transport in the advanced stress packages of \mf, including the Lake, Streamflow Routing, Multi-Aquifer Well and Unsaturated Zone Transport Packages.  The present implementation simulates solute advection between package features, such as between two stream reaches, but dispersive transport is not represented.  Likewise, solute transport between the advanced packages and the aquifer occurs only through advection.

\item The GWT Model has not yet been programmed to work with the Skeletal Storage, Compaction, and Subsidence (CSUB) Package for the GWF Model.  

\item There are many other differences between the \mf GWT Model and other solute transport models that work with MODFLOW, especially with regards to program design and input and output.  Descriptions for the GWT input and output are described here.

\end{enumerate}

\subsection{Units of Length and Time}
The GWF Model formulates the groundwater flow equation without using prescribed length and time units. Any consistent units of length and time can be used when specifying the input data for a simulation. This capability gives a certain amount of freedom to the user, but care must be exercised to avoid mixing units.  The program cannot detect the use of inconsistent units.

\subsection{Solute Mass Budget}
A summary of all inflow (sources) and outflow (sinks) of solute mass is called a mass budget.  \mf calculates a mass budget for the overall model as a check on the acceptability of the solution, and to provide a summary of the sources and sinks of mass to the flow system.  The solute mass budget is printed to the GWT Model Listing File for selected time steps.

\subsection{Time Stepping}

For the present implementation of the GWT Model, all terms in the solute transport equation are solved implicitly.  With the implicit approach applied to the transport equation, it is possible to take relatively large time steps and efficiently obtain a stable solution.  If the time steps are too large, however, accuracy of the model results will suffer, so there is usually some compromise required between the desired level of accuracy and length of the time step.  An assessment of accuracy can be performed by simply running simulations with shorter time steps and comparing results.

In \mf time step lengths are controlled by the user and specified in the Temporal Discretization (TDIS) input file.  When the flow model and transport model are included in the same simulation, then the length of the time step specified in TDIS is used for both models.  If the GWT Model runs in a separate simulation from the GWT Model, then the time steps used for the transport model can be different, and likely shorter, than the time steps used for the flow solution.  Instructions for specifying time steps are described in the TDIS section of this user guide; additional information on GWF and GWT configurations are in the Flow Model Interface section.  



\newpage
\subsection{GWT Model Name File}
\input{gwt/namefile.tex}

%\newpage
%\subsection{Structured Discretization (DIS) Input File}
%\input{gwf/dis}

%\newpage
%\subsection{Discretization with Vertices (DISV) Input File}
%\input{gwf/disv}

%\newpage
%\subsection{Unstructured Discretization (DISU) Input File}
%\input{gwf/disu}

\newpage
\subsection{Initial Conditions (IC) Package}
\input{gwt/ic}

\newpage
\subsection{Output Control (OC) Option}
\input{gwt/oc}

\newpage
\subsection{Observation (OBS) Utility for a GWT Model}
\input{gwt/gwt-obs}

\newpage
\subsection{Advection (ADV) Package}
\input{gwt/adv}

\newpage
\subsection{Dispersion (DSP) Package}
\input{gwt/dsp}

\newpage
\subsection{Source and Sink Mixing (SSM) Package}
Source and Sink Mixing (SSM) Package information is read from the file that is specified by ``SSM6'' as the file type.  Only one SSM Package can be specified for a GWT model.  The SSM Package is required if the flow model has any stress packages.

The SSM Package is used to add or remove solute mass from GWT model cells based on inflows and outflows from GWF stress packages.  If a GWF stress package provides flow into a model cell, that flow can be assigned a user-specified concentration.  If a GWT stress package removes water from a model cell, the concentration of that water is typically the concentration of the cell, but a ``MIXED'' option is also included so that the user can specify the concentration of that withdrawn water.  This may be useful for representing evapotranspiration, for example.  There are several different ways for the user to specify the concentrations.  

\begin{itemize}
\item The default condition is that sources have a concentration of zero and sinks withdraw water at the calculated concentration of the cell.  This default condition is assigned to any GWF stress package that is not included in a SOURCES block or FILEINPUT block.
\item A second option is to assign auxiliary variables in the GWF model and include a concentration for each stress boundary.  In this case, the user provides the name of the package and the name of the auxiliary variable containing concentration values for each boundary.  As described below for srctype, there are multiple options for defining this behavior.
\item A third option is to prepare an SPC6 file for any desired GWF stress package.  This SPC6 file allows users to change concentrations by stress period, or to use the time-series option to interpolate concentrations by time step.  This third option was introduced in MODFLOW version 6.3.0.  Information for this approach is entered in an optional FILEINPUT block below.  The SPC6 input file supports list-based concentration input for most corresponding GWF stress packages, but also supports a READASARRAYS array-based input format if a corresponding GWF recharge or evapotranspiration package uses the READASARRAYS option.
\end{itemize}

\noindent The auxiliary method and the SPC6 file input method can both be used for a GWT model, but only one approach can be assigned per GWF stress package.   If a flow package specified in the SOURCES or FILEINPUT blocks is also represented using an advanced transport package (SFT, LKT, MWT, or UZT), then the advanced transport package will override SSM calculations for that package.

\vspace{5mm}
\subsubsection{Structure of Blocks}
\lstinputlisting[style=blockdefinition]{./mf6ivar/tex/gwt-ssm-options.dat}
\lstinputlisting[style=blockdefinition]{./mf6ivar/tex/gwt-ssm-sources.dat}
\vspace{5mm}
\noindent \textit{FILEINPUT BLOCK IS OPTIONAL}
\lstinputlisting[style=blockdefinition]{./mf6ivar/tex/gwt-ssm-fileinput.dat}

\vspace{5mm}
\subsubsection{Explanation of Variables}
\begin{description}
% DO NOT MODIFY THIS FILE DIRECTLY.  IT IS CREATED BY mf6ivar.py 

\item \textbf{Block: OPTIONS}

\begin{description}
\item \texttt{PRINT\_FLOWS}---keyword to indicate that the list of SSM flow rates will be printed to the listing file for every stress period time step in which ``BUDGET PRINT'' is specified in Output Control.  If there is no Output Control option and ``PRINT\_FLOWS'' is specified, then flow rates are printed for the last time step of each stress period.

\item \texttt{SAVE\_FLOWS}---keyword to indicate that SSM flow terms will be written to the file specified with ``BUDGET FILEOUT'' in Output Control.

\end{description}
\item \textbf{Block: SOURCES}

\begin{description}
\item \texttt{pname}---name of the flow package for which an auxiliary variable contains a source concentration.  If this flow package is represented using an advanced transport package (SFT, LKT, MWT, or UZT), then the advanced transport package will override SSM terms specified here.

\item \texttt{srctype}---keyword indicating how concentration will be assigned for sources and sinks.  Keyword must be specified as either AUX or AUXMIXED.  For both options the user must provide an auxiliary variable in the corresponding flow package.  The auxiliary variable must have the same name as the AUXNAME value that follows.  If the AUX keyword is specified, then the auxiliary variable specified by the user will be assigned as the concenration value for groundwater sources (flows with a positive sign).  For negative flow rates (sinks), groundwater will be withdrawn from the cell at the simulated concentration of the cell.  The AUXMIXED option provides an alternative method for how to determine the concentration of sinks.  If the cell concentration is larger than the user-specified auxiliary concentration, then the concentration of groundwater withdrawn from the cell will be assigned as the user-specified concentration.  Alternatively, if the user-specified auxiliary concentration is larger than the cell concentration, then groundwater will be withdrawn at the cell concentration.  Thus, the AUXMIXED option is designed to work with the Evapotranspiration (EVT) and Recharge (RCH) Packages where water may be withdrawn at a concentration that is less than the cell concentration.

\item \texttt{auxname}---name of the auxiliary variable in the package PNAME.  This auxiliary variable must exist and be specified by the user in that package.  The values in this auxiliary variable will be used to set the concentration associated with the flows for that boundary package.

\end{description}
\item \textbf{Block: FILEINPUT}

\begin{description}
\item \texttt{pname}---name of the flow package for which an SPC6 input file contains a source concentration.  If this flow package is represented using an advanced transport package (SFT, LKT, MWT, or UZT), then the advanced transport package will override SSM terms specified here.

\item \texttt{SPC6}---keyword to specify that record corresponds to a source sink mixing input file.

\item \texttt{FILEIN}---keyword to specify that an input filename is expected next.

\item \texttt{spc6\_filename}---character string that defines the path and filename for the file containing source and sink input data for the flow package. The SPC6\_FILENAME file is a flexible input file that allows concentrations to be specified by stress period and with time series. Instructions for creating the SPC6\_FILENAME input file are provided in the next section on file input for boundary concentrations.

\item \texttt{MIXED}---keyword to specify that these stress package boundaries will have the mixed condition.  The MIXED condition is described in the SOURCES block for AUXMIXED.  The MIXED condition allows for water to be withdrawn at a concentration that is less than the cell concentration.  It is intended primarily for representing evapotranspiration.

\end{description}


\end{description}

\vspace{5mm}
\subsubsection{Example Input File}
\lstinputlisting[style=inputfile]{./mf6ivar/examples/gwt-ssm-example.dat}

% when obs are ready, they should go here

\newpage
\subsection{Stress Package Concentrations (SPC) -- List-Based Input}
As mentioned in the previous section on the SSM Package, concentrations can be specified for GWF stress packages using auxiliary variables, or they can be specified using input files dedicated to this purpose.  The Stress Package Concentrations (SPC) input file can be used to provide concentrations that are assigned for GWF sources and sinks.  An SPC input file can be list based or array based.  List-based input files can be used for list-based GWF stress packages, such as wells, drains, and rivers.  Array-based input files can be used for array-based GWF stress packages, such as recharge and evapotranspiration (provided the READASARRAYS options is used; these packages can also be provided in a list-based format).  Array-based SPC input files are discussed in the next section.  This section describes the list-based input format for the SPC input file.  

An SPC6 file can be prepared to provide user-specified concentrations for a GWF stress package, such a Well or General-Head Boundary Package, for example.  One SPC6 file applies to one GWF stress package.  Names for the SPC6 input files are provided in the FILEINPUT block of the SSM Package.  SPC6 entries cannot be specified in the GWT name file.  Use of the SPC6 input file is an alternative to specifying stress package concentrations as auxiliary variables in the flow model stress package.  

The boundary number in the PERIOD block corresponds to the boundary number in the GWF stress period package.  Assignment of the boundary number is straightforward for the advanced packages (SFR, LAK, MAW, and UZF) because the features in these advanced packages are defined once at the beginning of the simulation and they do not change.  For the other stress packages, however, the order of boundaries may change between stress periods.  Consider the following Well Package input file, for example:

\begin{verbatim}
# This is an example of a GWF Well Package
# in which the order of the wells changes from
# stress period 1 to 2.  This must be explicitly
# handled by the user if using the SPC6 input
# for a GWT model.
BEGIN options
END options

BEGIN dimensions
  MAXBOUND  3
  BOUNDNAMES
END dimensions

BEGIN period  1
  1 77 65   -2200  SHALLOW_WELL
  2 77 65   -24.0  INTERMEDIATE_WELL
  3 77 65   -6.20  DEEP_WELL
END period

BEGIN period  2
  1 77 65   -1100  SHALLOW_WELL
  3 77 65   -3.10  DEEP_WELL
  2 77 65   -12.0  INTERMEDIATE_WELL
END period
\end{verbatim}

\noindent In this Well input file, the order of the wells changed between periods 1 and 2.  This reordering must be explicitly taken into account by the user when creating an SSMI6 file, because the boundary number in the SSMI file corresponds to the boundary number in the Well input file.  In stress period 1, boundary number 2 is the INTERMEDIATE\_WELL, whereas in stress period 2, boundary number 2 is the DEEP\_WELL.  When using this SSMI capability to specify boundary concentrations, it is recommended that users write the corresponding GWF stress packages using the same number, cell locations, and order of boundary conditions for each stress period.   In addition, users can activate the PRINT\_FLOWS option in the SSM input file.  When the SSM Package prints the individual solute flows to the transport list file, it includes a column containing the boundary concentration.  Users can check the boundary concentrations in this output to verify that they are assigned as intended.

\vspace{5mm}
\subsubsection{Structure of Blocks}
\vspace{5mm}

\noindent \textit{FOR EACH SIMULATION}
\lstinputlisting[style=blockdefinition]{./mf6ivar/tex/utl-spc-options.dat}
\lstinputlisting[style=blockdefinition]{./mf6ivar/tex/utl-spc-dimensions.dat}
\vspace{5mm}
\noindent \textit{FOR ANY STRESS PERIOD}
\lstinputlisting[style=blockdefinition]{./mf6ivar/tex/utl-spc-period.dat}

\vspace{5mm}
\subsubsection{Explanation of Variables}
\begin{description}
% DO NOT MODIFY THIS FILE DIRECTLY.  IT IS CREATED BY mf6ivar.py 

\item \textbf{Block: OPTIONS}

\begin{description}
\item \texttt{PRINT\_INPUT}---keyword to indicate that the list of spc information will be written to the listing file immediately after it is read.

\item \texttt{TS6}---keyword to specify that record corresponds to a time-series file.

\item \texttt{FILEIN}---keyword to specify that an input filename is expected next.

\item \texttt{ts6\_filename}---defines a time-series file defining time series that can be used to assign time-varying values. See the ``Time-Variable Input'' section for instructions on using the time-series capability.

\end{description}
\item \textbf{Block: DIMENSIONS}

\begin{description}
\item \texttt{maxbound}---integer value specifying the maximum number of spc cells that will be specified for use during any stress period.

\end{description}
\item \textbf{Block: PERIOD}

\begin{description}
\item \texttt{iper}---integer value specifying the starting stress period number for which the data specified in the PERIOD block apply.  IPER must be less than or equal to NPER in the TDIS Package and greater than zero.  The IPER value assigned to a stress period block must be greater than the IPER value assigned for the previous PERIOD block.  The information specified in the PERIOD block will continue to apply for all subsequent stress periods, unless the program encounters another PERIOD block.

\item \texttt{bndno}---integer value that defines the boundary package feature number associated with the specified PERIOD data on the line. BNDNO must be greater than zero and less than or equal to MAXBOUND.

\item \texttt{spcsetting}---line of information that is parsed into a keyword and values.  Keyword values that can be used to start the MAWSETTING string include: CONCENTRATION.

\begin{lstlisting}[style=blockdefinition]
CONCENTRATION <@concentration@>
\end{lstlisting}

\item \textcolor{blue}{\texttt{concentration}---is the boundary concentration. If the Options block includes a TIMESERIESFILE entry (see the ``Time-Variable Input'' section), values can be obtained from a time series by entering the time-series name in place of a numeric value. By default, the CONCENTRATION for each boundary feature is zero.}

\end{description}


\end{description}

\subsubsection{Example Input File}
\lstinputlisting[style=inputfile]{./mf6ivar/examples/utl-spc-example.dat}

% SPC array based
\newpage
\subsection{Stress Package Concentrations (SPC) -- Array-Based Input}

This section describes array-based input for the SPC input file.  If the READASARRAYS options is specified for either the GWF Recharge (RCH) or Evapotranspiration (EVT) Packages, then concentrations for these packages can be specified using array-based concentration input.  This SPC array-based input is distinguished from the list-based input in the previous section through specification of the READASARRAYS option.  When the READASARRAYS option is specified, then there is no DIMENSIONS block in the SPC input file.  Instead, the shape of the array for concentrations is the number of rows by number of columns (NROW, NCOL), for a regular MODFLOW grid (DIS), and the number of cells in a layer (NCPL) for a discretization by vertices (DISV) grid.

\vspace{5mm}
\subsubsection{Structure of Blocks}
\vspace{5mm}

\noindent \textit{FOR EACH SIMULATION}
\lstinputlisting[style=blockdefinition]{./mf6ivar/tex/utl-spca-options.dat}
\vspace{5mm}
\noindent \textit{FOR ANY STRESS PERIOD}
\lstinputlisting[style=blockdefinition]{./mf6ivar/tex/utl-spca-period.dat}

\vspace{5mm}
\subsubsection{Explanation of Variables}
\begin{description}
% DO NOT MODIFY THIS FILE DIRECTLY.  IT IS CREATED BY mf6ivar.py 

\item \textbf{Block: OPTIONS}

\begin{description}
\item \texttt{READASARRAYS}---indicates that array-based input will be used for the SPC Package.  This keyword must be specified to use array-based input.

\item \texttt{PRINT\_INPUT}---keyword to indicate that the list of spc information will be written to the listing file immediately after it is read.

\item \texttt{TAS6}---keyword to specify that record corresponds to a time-array-series file.

\item \texttt{FILEIN}---keyword to specify that an input filename is expected next.

\item \texttt{tas6\_filename}---defines a time-array-series file defining a time-array series that can be used to assign time-varying values. See the Time-Variable Input section for instructions on using the time-array series capability.

\end{description}
\item \textbf{Block: PERIOD}

\begin{description}
\item \texttt{iper}---integer value specifying the starting stress period number for which the data specified in the PERIOD block apply.  IPER must be less than or equal to NPER in the TDIS Package and greater than zero.  The IPER value assigned to a stress period block must be greater than the IPER value assigned for the previous PERIOD block.  The information specified in the PERIOD block will continue to apply for all subsequent stress periods, unless the program encounters another PERIOD block.

\item \texttt{concentration}---is the concentration of the associated Recharge or Evapotranspiration stress package.  The concentration array may be defined by a time-array series (see the "Using Time-Array Series in a Package" section).

\end{description}


\end{description}

\subsubsection{Example Input File}
\lstinputlisting[style=inputfile]{./mf6ivar/examples/utl-spca-example.dat}



\newpage
\subsection{Mobile Storage and Transfer (MST) Package}
\input{gwt/mst}

\newpage
\subsection{Immobile Storage and Transfer (IST) Package}
\input{gwt/ist}

\newpage
\subsection{Constant Concentration (CNC) Package}
\input{gwt/cnc}

\newpage
\subsection{Mass Source Loading (SRC) Package}
\input{gwt/src}

\newpage
\subsection{Streamflow Transport (SFT) Package}
\input{gwt/sft}

\newpage
\subsection{Lake Transport (LKT) Package}
\input{gwt/lkt}

\newpage
\subsection{Multi-Aquifer Well Transport (MWT) Package}
\input{gwt/mwt}

\newpage
\subsection{Unsaturated Zone Transport (UZT) Package}
\input{gwt/uzt}

\newpage
\subsection{Flow Model Interface (FMI) Package}
Flow Model Interface (FMI) Package information is read from the file that is specified by ``FMI6'' as the file type.  The FMI Package is optional, but if provided, only one FMI Package can be specified for a GWT model.

For most simulations, the GWT Model needs groundwater flows for every cell in the model grid, for all boundary conditions, and for other terms, such as the flow of water in or out of storage.  The FMI Package is the interface between the GWT Model and simulated groundwater flows provided by a corresponding GWF Model that is running concurrently within the simulation or from binary budget files that were created from a previous GWF model run.  The following are several different FMI simulation cases:

\begin{itemize}

\item Flows are provided by a corresponding GWF Model running in the same simulation---in this case, all groundwater flows are calculated by the corresponding GWF Model and provided through FMI to the transport model.  This is a common use case in which the user wants to run the flow and transport models as part of a single simulation.  The GWF and GWT models must be part of a GWF-GWT Exchange that is listed in mfsim.nam.  If a GWF-GWT Exchange is specified by the user, then the user does not need to specify an FMI Package input file for the simulation, unless an FMI option is needed.  If a GWF-GWT Exchange is specified and the FMI Package is specified, then the PACKAGEDATA block below is not read or used.

\item There is no groundwater flow and the user is interested only in the effects of diffusion, sorption, and decay or production---in this case, FMI should not be provided in the GWT name file and the GWT model should not be listed in any GWF-GWT Exchanges in mfsim.nam.  In this case, all groundwater flows are assumed to be zero and cells are assumed to be fully saturated.  The SSM Package should not be activated in this case, because there can be no sources or sinks of water.  Likewise, none of the advanced transport packages (LKT, SFT, MWT, and UZT) should be specified in the GWT name file.  This type of model simulation without an FMI Package is included as an option to represent diffusion, sorption, and decay or growth in the absence of any groundwater flow.

\item Flows are provided from a previous GWF model simulation---in this case FMI should be provided in the GWT name file and the head and budget files should be listed in the FMI PACKAGEDATA block.  In this case, FMI reads the simulated head and flows from these files and makes them available to the transport model.  There are some additional considerations when the heads and flows are provided from binary files.

\begin{itemize}
\item The binary budget file must contain the simulated flows for all of the packages that were included in the GWF model run.  Saving of flows can be activated for all packages by specifying ``SAVE\_FLOWS'' as an option in the GWF name file.  The GWF Output Control Package must also have ``SAVE BUGET ALL'' specified.  The easiest way to ensure that all flows and heads are saved is to use the following simple form of a GWF Output Control file:

\begin{verbatim}
BEGIN OPTIONS
  HEAD FILEOUT mymodel.hds
  BUDGET FILEOUT mymodel.bud
END OPTIONS

BEGIN PERIOD 1
  SAVE HEAD ALL
  SAVE BUDGET ALL
END PERIOD
\end{verbatim}

\item The binary budget file must have the same number of budget terms listed for each time step.  This will always be the case when the binary budget file is created by \mf.
\item The advanced flow packages (LAK, SFR, MAW, and UZF) all have options for saving a detailed budget file the describes all of the flows for each lake, reach, well, or UZF cell.  These budget files can also be used as input to FMI if a corresponding advanced transport package is needed, such as LKT, SFT, MWT, and UZT.  If the Water Mover Package is also specified for the GWF Model, then the the budget file for the Water Mover Package will also need to be specified as input to this FMI Package.
\item The binary heads file must have heads saved for all layers in the model.  This will always be the case when the binary head file is created by \mf.  This was not always the case as previous MODFLOW versions allowed different save options for each layer.
\item If the binary budget and head files have more than one time step for a single stress period, then the budget and head information must be contained within the binary file for every time step in the simulation stress period.
\item The binary budget and head files must correspond in terms of information stored for each time step and stress period.
\item If the binary budget and head files have information provided for only the first time step of a given stress period, this information will be used for all time steps in that stress period in the GWT simulation. If the final stress period (which may be the only stress period) in the binary budget and head files has information provided for only one step step, this information will be used for any subsequent time steps and stress periods in the GWT simulation. This makes it possible to provide flows, for example, from a steady-state GWF stress period and have those flows used for all GWT time steps in that stress period, for all remaining time steps in the GWT simulation, or for all time steps throughout the entire GWT simulation. With this option, it is possible to have smaller time steps in the GWT simulation than the time steps used in the GWF simulation. Note that this cannot be done when the GWF and GWT models are run in the same simulation, because in that case, both models are solved for each time step in the stress period, as listed in the TDIS Package. This option for reading flows from a previous GWF simulation may offer an efficient alternative to running both models in the same simulation, but it comes at the cost of having potentially very large budget files.
\end{itemize}

\end{itemize}

\noindent Determination of which FMI use case to invoke requires careful consideration of the different advantages and disadvantages of each case.  For example, running GWT and GWF in the same simulation can often be faster because GWF flows are passed through memory to the GWT model instead of being written to files.  The disadvantage of this approach is that the same time step lengths must be used for both GWF and GWT.  Ultimately, it should be relatively straightforward to test different ways in which GWF and GWT interact and select the use case most appropriate for the particular problem. 

\vspace{5mm}
\subsubsection{Structure of Blocks}
\lstinputlisting[style=blockdefinition]{./mf6ivar/tex/gwt-fmi-options.dat}
\lstinputlisting[style=blockdefinition]{./mf6ivar/tex/gwt-fmi-packagedata.dat}

\vspace{5mm}
\subsubsection{Explanation of Variables}
\begin{description}
\input{./mf6ivar/tex/gwt-fmi-desc.tex}
\end{description}

\vspace{5mm}
\subsubsection{Example Input File}
\lstinputlisting[style=inputfile]{./mf6ivar/examples/gwt-fmi-example.dat}



\newpage
\subsection{Mover Transport (MVT) Package}
\input{gwt/mvt}

